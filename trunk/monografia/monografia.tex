\documentclass[bacharelado]{unb-cic}
\usepackage[american,brazil]{babel}
\usepackage[T1]{fontenc}
\usepackage{indentfirst}
\usepackage{natbib}
\usepackage{xcolor,graphicx,url}
\usepackage[latin1]{inputenc}
\usepackage{listings}
\usepackage{subfigure}
\setlength{\parindent}{16pt}
\setlength{\parskip}{2ex}

\bibpunct[; ]{(}{)}{,}{a}{}{;} %muda colchetes para parenteses

% definicoes previas do documento
\title{Deskworld: Software Simulador de F\'isica 2D para Mesas com Superf\'icie Multi-Toque}

\orientador[a]{\prof[a] \dr[a] Carla Denise Castanho}{CIC/UnB}
\coorientador{\prof \dr Marcus Vinicius Lamar}{CIC/UnB}

\coordenador{\prof \dr Marcus Vinicius Lamar}{CIC/UnB}

\diamesano{08}{fevereiro}{2011}

%\membrobanca{\prof\dr Marcus Vinicius Lamar}{CIC/UnB}

\membrobanca{\prof\dr Pedro de Azevedo Berger}{CIC/UnB}

\membrobanca{\prof\dr Ricardo Pezzuol Jacobi}{CIC/UnB}

\autor{Danilo Gaby Andersen}{Trindade}
\coautor{Victor Sampaio}{Zucca}
\CDU{004.4}

\palavraschave{multi-toque, f\'isica, jogo }
\keywords{multi-touch, physics, game}


\begin{document}

\maketitle

\pretextual
\begin{agradecimentos}
Gostariamos de agradecer primeiramente aos membros da banca
\end{agradecimentos}

\begin{resumo}
Este trabalho foca na contru\c{c}\~ao de um jogo interativo para v\'arios jogadores em superf\' icies com telas sens\'iveis a v\'arios toques simult\^aneos.
\end{resumo}

\selectlanguage{american}

\begin{abstract}
This paper focus on the construction of an interactive multiplayer game for multi-touch surfaces.
\end{abstract}

\selectlanguage{brazil}
\tableofcontents
\listoffigures
\listoftables

\textual

\chapter{Introdu\c{c}\~ao}
\label{cap1}

A ind�stria do entretenimento � uma das que mais cresce, e o Brasil est� entre as que apresentam maior crescimento nessa �rea~[\citenum{sebrae}]. Um dos motivos � a grande variedade do que pode ser feito para essa ind�stria. Por exemplo, se for entregue uma caixa de papel�o a uma crian�a, ela ir� fazer um forte ou uma espa�onave e se divertir� com aquilo. Isso faz com que essa ind�stria seja altamente lucrativa e muito buscada. Em~[\citenum{retro}] e~[\citenum{I2GMDVP}], pode-se ver a hist�ria dos primeiros \textit{video games}, explicadas a seguir. A ind�stria de \textit{video games} come�ou somente com alguns programas demonstrativos para computadores, a partir de 1952. Foram criados pequenos jogos que utilizavam os inputs limitados dispon�veis, como um jogo-da-velha eletr�nico, chamado \textit{OXO} (Figura~[\ref{retro1}],) criado por \textit{Alexander S. Douglas}, ou o \textit{Tennis for two} (Figura~[\ref{retro2}]), criado pelo \textit{Brookhaven National Laboratory} (Departamento de energia dos EUA).

\begin{figure}[h!]
	\begin{center}
	\includegraphics[scale=0.4]{oxo.png}
	\caption{Jogo para \textit{EDSAC} \textit{OXO}~[\citenum{retro}]}
	\label{retro1}
	\end{center}
\end{figure}

\begin{figure}[h!]
	\begin{center}
	\includegraphics[scale=0.4]{tennis_for_two.jpg}
	\caption{Jogo para um oscilosc�pio \textit{Tennis for two}~[\citenum{retro}]}
	\label{retro2}
	\end{center}
\end{figure}

J� em 1962, \textit{Steve Russell} desenvolveu um jogo chamado \textit{Spacewar!} (Figura~[\ref{retro3}]), inicialmente para um sistema chamado \textit{PDP-1}. Ele consiste de um jogo de dois jogadores, cada um no controle de uma nave, em que o objetivo � atirar m�sseis para destruir a nave do oponente. O jogo conseguiu tanta popularidade que em seguida foi adaptado para v�rios outros sistemas populares na �poca, e o jogo era um teste de sistema t�o bom para o \textit{PDP-1} que o sistema era vendido j� com o jogo em mem�ria. Isso fez com que o \textit{Spacewar!} fosse o primeiro jogo de computador altamente difundido.

\begin{figure}[h!]
	\begin{center}
	\includegraphics[scale=0.6]{spacewar.png}
	\caption{Jogo para computadores antigos \textit{Spacewars!}~[\citenum{retro}]}
	\label{retro3}
	\end{center}
\end{figure}

O primeiro console dom�stico lan�ado foi o \textit{Magnavox Odyssey}~[\citenum{odyssey}], que pode ser observado na Figura~[\ref{odyssey}], lan�ado em 1972. ele inaugurou uma id�ia inovadora, de se criar um aparelho computacional independente de uma tela, para ser conectado a televisores, o que at� ent�o n�o havia sido pensado, os computadores sempre vinham acoplados a suas telas. Al�m disso, ele tamb�m inaugurou outro conceito que � o de m�dia remov�vel para os consoles, que evolui para os cartuchos e, hoje em dia, aos DVDs e Blu-Rays. Por�m, devido ao pre�o um pouco caro para a �poca e uma m� estrat�gia de marketing, o \textit{Odyssey} n�o foi muito popular. Uma outra companhia que inaugurou seu console em 1976, a \textit{Atari}, teve uma melhor estrat�gia. Ela lan�ou seu nome no mercado devido ao seu jogo \textit{Pong}, um simples jogo de t�nis, que foi inicialmente lan�ado para uma m�quina com uma tela que permitia o jogo atrav�s de um sistema de moedas em 1972 e foi adaptado para uma vers�o dom�stica em 1973~[\citenum{atari}]. Em 1976, utilizando um novo processador barato na �poca, o \textit{MOS 6502}, a \textit{Atari} lan�a seu primeiro console dom�stico com possiblidade de trocar jogos, o \textit{Atari 2600} (Figura~[\ref{atari}]). Esse console teve um enorme sucesso de acordo com~[\citenum{atari}].

\begin{figure}[h!]
	\begin{center}
	\includegraphics[scale=0.4]{odyssey.jpg}
	\caption{Console \textit{Odyssey} da \textit{Magnavox}~[\citenum{odyssey}]}
	\label{odyssey}
	\end{center}
\end{figure}

\begin{figure}[h!]
	\begin{center}
	\includegraphics[scale=1.0]{atari.jpg}
	\caption{Console \textit{Atari 2600} da \textit{Atari}~[\citenum{atari}]}
	\label{atari}
	\end{center}
\end{figure}

Ambos esses consoles dom�sticos tinham controladores para se entrar o \textit{input} do jogo, o do \textit{Odyssey} sendo composto de duas rodas e um bot�o \textit{reset}, e o do \textit{Atari} sendo composto de um \textit{joystick}, que � uma haste que pode ser movida para enviar dire��es ao jogo, e um bot�o.

Ap�s esse per�odo, os \textit{video games} passaram por um decl�nio, e somente por 1980 que a \textit{Nintendo} resgatou esse mercado com o lan�amento de um jogo conhecido como \textit{Donkey Kong}, inicialmente desenvolvido em conjunto com a \textit{Atari}, por�m uma diverg�ncia sobre direitos autorais do \textit{Donkey Kong} fez as companhias terminarem a parceria~[\citenum{atari}]. Em 1985 a \textit{Nintendo} lan�ou seu console, o \textit{NES} (Figura~[\ref{nes}]), e passou a dominar o mercado, com pouca competi��o de sua nova concorrente, \textit{SEGA}, e seu console \textit{Master System}. Esses consoles possuiam \textit{gamepads} para a movimenta��o nos jogos, sendo compostos de uma cruz direcional que funciona similar ao joystick e v�rios bot�es.

\begin{figure}[h!]
	\begin{center}
	\includegraphics[scale=0.4]{nes.jpg}
	\caption{Console \textit{NES} da \textit{Nintendo}~[\citenum{nes}]}
	\label{nes}
	\end{center}
\end{figure}

Nesse momento a ind�stria do entretenimento eletr�nico teve um grande crescimento, abrindo o mercado para cada vez mais competidores a medida que se foi evoluindo as gera��es de consoles. A \textit{Nintendo} se manteve forte por todo esse tempo, se tornando uma das tr�s companhias que dominam esse mercado hoje em dia. Sua principal competidora, a \textit{SEGA}, n�o conseguiu acompanhar, e seu �ltimo console, o \textit{Dreamcast}, teve pouca venda. Atualmente a \textit{SEGA} ainda � uma empresa grande, entretanto, s� atua na produ��o de jogos para outros consoles.

Durante a evolu��o dos jogos, uma empresa fez parceria com a \textit{Nintendo} para tentar lan�ar um novo sistema em que os cartuchos foram trocados por CDs, diminuindo seu custo e aumenta a sua capacidade de armazenamento. Essa empresa � a \textit{Sony}. Devido a desentendimentos a parceria terminou, por�m, a \textit{Sony} insistiu na id�ia e em 1994 lan�ou o seu console, o \textit{Playstation}. Esse console foi um sucesso de vendas enorme, no entanto, o que realmente colocou a \textit{Sony} no mapa dos consoles foi o seu pr�ximo console, \textit{Playstation 2}. Esse console foi uma grande revolu��o no seu tempo e saiu um ano antes dos consoles de sua gera��o, o que ocasionou no maior �ndice de vendas de um console de \textit{video game} de todos os tempos, vendendo mais de 140 milh�es de unidades. Atualmente a \textit{Sony} compete no mercado como uma das maiores empresas de divers�o eletr�nica.

Outra empresa que resolveu explorar esse mundo dos consoles � a \textit{Microsoft}. Gigante no mundo da computa��o, ela viu a oportunidade de aplicar seus conhecimentos para criar um sistema barato, eficiente e inovador. Em 2001, ela lan�ou seu primeiro console, o \textit{Xbox}. Esse console, que � da mesma gera��o que o \textit{Playstation 2}, saiu um pouco tarde, afetando bastante as suas vendas. Ele foi feito baseado na arquitetura de um PC tradicional, apesar de possuir uma produ��o muito cara, ele acabou dando altas perdas iniciais para a empresa. Por�m, a \textit{Microsoft} insistiu nesse mercado e hoje compete com as outras duas empresas por essa gera��o de consoles.

Todos esses consoles utilizaram o mesmo m�todo de input de seus predecessores, com o uso do \textit{gamepad} como entrada principal de \textit{input}, alguns possuindo outros controladores, como pistolas que podiam ser apontadas para a tela. Entretanto, devido ao alto pre�o desses controladores e baixa quantia de jogos compat�veis, nenhum deles chegou a fazer alguma marca no mercado.

Atualmente, o mercado dos jogos eletr�nicos conta com tr�s consoles competindo pelo mercado, o \textit{Nintendo Wii} da \textit{Nintendo}, o \textit{Xbox 360} da \textit{Microsoft} e o \textit{Playstation 3} da \textit{Sony}, os tr�s podem ser observados na Figura~[\ref{consoles}], al�m dos tr�s estarem competindo com jogos desenvolvido para os computadores pessoais, que hoje em dia existem em muitas resid�ncias, e aparelhos dos mais variados como celulares com superf�cie sens�vel ao toque. Nesse mercado de competi��o, para ser levado em considera��o, deve-se ser inovador.

\begin{figure}[h!]
	\begin{center}
	\includegraphics[scale=1.4]{wii-ps3-xbox360.jpg}
	\caption{Consoles de �ltima gera��o. Da esquerda pra direita: \textit{Wii}, \textit{Playstation 3} e \textit{Xbox 360}~[\citenum{consoles}]}
	\label{consoles}
	\end{center}
\end{figure}

Durante a evolu��o natural da intera\c{c}\~ao entre o homem e a m\'aquina, busca-se cada vez mais a adapta\c{c}\~ao dos gestos para o ambiente virtual. Ao fornecer ao usu\'ario gestos f�ceis de usar, diminui a curva de aprendizado de sua utiliza��o, bem como mant\^em um ambiente mais familiar ao usu\'ario. As grandes companhias competidoras j� est�o caminhando em dire��o a isso. O \textit{Nintendo Wii} possui controles sens�veis ao movimento, permitindo o uso de gestos para o controle em seus jogos, o \textit{Wiimote} (Figura~[\ref{wiimote}]), assim como o \textit{Playstation 3} possui seus controles sens�veis ao movimento e com no��o de profundidade, o \textit{Playstation Move} (Figura~[\ref{move}]), e o \textit{Xbox 360} possui o \textit{Kinect} (Figura~[\ref{kinect}]), uma c�mera inteligente capaz de identificar gestos.

\begin{figure}[h!]
	\begin{center}
	\includegraphics[scale=0.2]{wiimote.jpg}
	\caption{Controle para \textit{Nintendo Wii} \textit{Wiimote}~[\citenum{controles}]}
	\label{wiimote}
	\end{center}
\end{figure}

\begin{figure}[h!]
	\begin{center}
	\includegraphics[scale=0.2]{move.jpg}
	\caption{Controle para \textit{Playstation 3} \textit{Playstation Move}~[\citenum{controles}]}
	\label{move}
	\end{center}
\end{figure}

\begin{figure}[h!]
	\begin{center}
	\includegraphics[scale=0.2]{kinect.jpg}
	\caption{C�mera para \textit{Xbox 360} \textit{Kinect}~[\citenum{controles}]}
	\label{kinect}
	\end{center}
\end{figure}

Como apresentado anteriormente, os consoles principais disputam o mercado de jogos com v�rios aparelhos diferentes. Um desses aparelhos que ainda � pouco explorado trata de mesas com superf�cie multi-toque. Tem poucos exemplos de mesas multi-toque, como a \textit{Reactable}~[\citenum{react}] ou a \textit{Microsoft Surface}~[\citenum{microsoftsurface}], e nenhuma que j� esteja popularmente dispersada no mercado, devido ao alto custo que as op��es comerciais normalmente possuem. Esta interface, no entanto, possui um grande potencial, e existem alternativas para construir mesas deste tipo usando materiais baratos, o que foi feito no decorrer deste trabalho.

Estas mesas multi-toque j� possuem alguns \textit{softwares} dispon�veis, por�m n�o existe um bom software que demonstre inteiramente as capacidades de \textit{input} da mesa. Dessa forma, este trabalho mostra a cria��o de um software simulador de f�sica bidimensional para uma mesa com superf�cie multi-toque, que devido as suas propriedades possui uma boa sinergia com o m�todo de interface escolhido, explorando assim, as capacidades de \textit{input} da mesa em um �nico \textit{software} de estilo pouco explorado e divertido.

Os softwares simuladores de f�sica s�o essencialmente programas que come�am com um mundo em aberto, e o usu�rio pode desenhar nele os objetos que desejar, podendo atribuir a eles as propriedades que quiser, e eles ir�o interagir de acordo com as leis da f�sica, obedecendo gravidade, acelera��o e empuxo. Assim, um software nesse estilo permitiria utilizar as capacidades da mesa demonstrando seus gestos dispon�veis, como scale e rotate, bem como a detec��o de gestos padr�es como o desenho de um quadrado ou c�rculo. Softwares nesse estilo incluem o \textit{Phun}, \textit{Crayon Physics} e \textit{Numpty Physics}, este �ltimo que possui uma convers�o para \textit{input} multi-toque, por�m ele possui recursos limitados e n�o foi planejado para corretamente suportar um n�mero arbrit�rio de toques vindo de um n�mero desconhecido de usu�rios, o que deve ser cuidado ao desenvolver softwares para uma mesa multi-toque.

O software desenvolvido neste trabalho, chamado de \textit{Deskworld}, segue estas mesmas regras de softwares simuladores de fisica, por�m, como � planejado para mesas multi-toque, possui algumas propriedades novas, como a possibilidade de dividir o mundo para utiliza��o de usu�rios em condi��es diferentes, e a detec��o de gestos para aproximar formas ao seu estado ideal, pois � dif�cil de desenhar precisamente utilizando somente os dedos.

Como as mesas com superf�cie multi-toque n�o s�o ainda dispon�veis em massa para a resid�ncia dos usu�rios, sua maior taxa de uso est� focada em exposi��es, onde os usu�rios possuem somente um encontro breve com a mesa. Assim, um aplicativo desenvolvido para uma mesa dessas necessita ser f�cil de controlar e possuir suporte a m�ltiplos usu�rios a interagindo simult�neamente a qualquer momento, o que � feito com o software desenvolvido neste trabalho. Ao mesmo tempo, este software possui uma grande capacidade de cria��o, proporcionando tamb�m divers�o para usu�rios a longo prazo.

O restante deste trabalho est� dividido conforme segue. O segundo cap�tulo deste trabalho apresenta trabalhos correlatos, exibindo mesas parecidas a desenvolvida e \textit{softwares} similares ao proposto, ambos para mesas como para outros sistemas em geral, sens�veis ao toque ou n�o. O terceiro cap�tulo apresenta o fundamento te�rico por tr�s da mesa, explicando os princ�pios de processamento de imagem necess�rios para entender toques na mesa, bem como os filtros utilizados, e explica os m�todos de ilumina��o e como isso afeta este projeto. O quarto cap�tulo detalha sobre a constru��o da mesa em si, onde est� exposto o projeto da mesa, explicando as decis�es de \textit{hardware} escolhidos de acordo com as dimens�es da mesa e explicando como foram feitos fisicamente os passos necess�rios para a detec��o de toque, com a utiliza��o de LEDs e uma c�mera infravermelha. O quinto cap�tulo demonstra a implementa��o do \textit{software} proposto, chamado \textit{Deskworld}, apresentando aspectos como o \textit{Design} que explica seus detalhes, conceito, objetos e regras. Neste mesmo cap�tulo � exibido a arquitetura utilizada, detalhes de implementa��o, mostrando como foi implementado em uma vis�o de n�vel mais baixo, explicando as decis�es do projeto e as escolhas feitas sobre a orienta��o a componentes. Na se��o final deste cap�tulo, explica-se com mais detalhes sobre as ferramentas utilizadas de suporte para a implementa��o do jogo proposto, como a \textit{Community Core Vision} e a engine \textit{Box2D}. Por fim, o sexto cap�tulo elenca algumas considera��es finais e trabalhos futuros.
\chapter{Trabalhos Correlatos}
\label{cap2}
Neste cap�tulo, primeiro ser� exemplificado mesas com superf�cie multi-toque similares a utilizada neste trabalho, seguido de exemplos de softwares interativos, como jogos, que j� existem para mesas com superf�cie multi-toque, terminando com exemplos de softwares simuladores de f�sica em geral para sistemas diversos, incluindo para as mesas com superf�cie multi-toque.

\section{Mesas com superf\'icies multi-toque}
\label{cap2.1}
As mesas com tela multi-toque s\~ao relativamente recentes no mercado, com muitas pesquisas ainda sendo desenvolvidas para viabiliz�-las ao p\'ublico. Uma alternativa criativa foi proposta por~[\citenum{mtmini}] onde demonstra a viabilidade de se montar uma pequena mesa com superf�cie multi-toque com menos de US\$50 d�lares. Com materiais baratos como papel�o, papel e vidro pode-se construir uma mesa utilizando um computador pessoal e uma \textit{webcam}.

%TODO estudo sobre viabilidade, eh bom botar uma referencia disso aqui
Utilidades diversas s\~ao propostas para essas mesas. Tem-se por exemplo a \textit{Reactable}~[\citenum{react}], Figura~\ref{react}, dispon\'ivel desde 2008, que, apesar de n�o exatamente possibilitar o toque do usu�rio, trata-se de uma mesa que utiliza marcadores fiduciais para habilitar a intera\c{c}\~ao de seus usu\'arios com a m\'usica, fornecendo um instrumento musical diferente. Ela \'e utilizada em concertos e eventos.

\begin{figure}[h!]
	\begin{center}
	\includegraphics[scale=0.60]{reactable.jpg}
	\caption{Mesa com marcadores fiduciais \textit{Reactable}~[\citenum{react}]}
	\label{react}
	\end{center}
\end{figure}

Pode-se citar tamb\'em a \textit{Microsoft Surface}~[\citenum{microsoftsurface}], Figura~\ref{msurface}, que proporciona funcionalidades diversas como habilidade de compartilhamento de Figuras e arquivos atrav\'es da intera\c{c}\~ao com objetos eletr\^onicos colocados sobre sua superf\'icie e comandos via o toque do usu�rio. 

\begin{figure}[h!]
	\begin{center}
	\includegraphics[scale=0.60]{microsoft_surface_main.jpg}
	\caption{Mesa \textit{Microsoft Surface}~[\citenum{microsoftsurface}]}
	\label{msurface}
	\end{center}
\end{figure}

Essas duas mesas s�o as mais popularmente conhecidas, apesar de existirem outros projetos desenvolvidos para exibi��es, alguns que podem ser encomendados tamb�m. Devido ao seu alto custo, muitas outras alternativas de constru��es artesanais de mesas foram propostas. Podemos citar algumas, como a constru�da durante o desenvolvimento do jogo \textit{Eco-Defense} em~[\citenum{pedrosaulo}], a feita em~[\citenum{IRTaktiks}] junto do jogo \textit{IRTaktiks}, bem como a \textit{Virttable}~[\citenum{virttable}].

A id�ia de todas � a mesma: Temos uma superf�cie onde se usa um projetor multim�dia para se projetar a tela na superf�cie da mesa, e o usu�rio pode tocar a superf�cie para interagir com o software. A superf�cie � iluminada com a utiliza��o de leds e os toques s�o detectados por uma c�mera sem filtro infravermelho que os interpreta em rela��o a posi��o na mesa.

\begin{figure}[h!]
	\begin{center}
	\includegraphics[scale=0.40]{touch_table.jpg}
	\caption{Projeto de mesa multi-toque com DI~[\citenum{scient}]}
	\label{touchtable}
	\end{center}
\end{figure}

Existem v�rias maneiras de se iluminar a mesa. DI (\textit{Diffuse Ilumination}), Figura~\ref{touchtable}, se trata de LEDs iluminando a mesa por baixo do acr�lico. Esse m�todo � utilizado, por exemplo, na \textit{Microsoft Surface}~[\citenum{microsoftsurface}] e na mesa constru�da durante o desenvolvimento do jogo \textit{Eco-Defense}~[\citenum{pedrosaulo}], por exemplo.

Outro sistema de ilumina��o � o sistema FTIR (\textit{Frustrated Total Internal Reflection}), Figura~\ref{FTIR}. Nesse sistema, os LEDS s�o imbutidos no acr�lico da superf�cie da mesa. Assim, a ilumina��o sofre de um fen�meno denominado reflex�o interna total frustada. Isso significa que a ilumina��o fica retida dentro do acr�lico sem conseguir sair. Ao momento que um dedo toca na superf�cie, a capacidade de refra��o da superf�cie muda, o que permite que a luz saia e ilumine o toque, e uma cam�ra infravermelho consegue captar essa luz, detectando o local do toque. Essa detec��o � normalmente bem mais precisa que a detec��o por DI, por possuir menos interfer�ncia do fundo, por�m ela n�o funciona muito bem para objetos que n�o s�o o corpo humano, sendo ruins para funcionamento com marcadores fiduciais.

\begin{figure}[h!]
	\begin{center}
	\includegraphics[scale=0.50]{touch_FTIR_port.jpg}
	\caption{Esquema de detec��o de toque FTIR (adaptada de~[\citenum{scient}])}
	\label{FTIR}
	\end{center}
\end{figure}

A �ltima abordagem trata de uma mesclagem das duas abordagens anteriores, como vemos no exemplo da \textit{Virttable}~[\citenum{virttable}], Figura~\ref{virttable}, que utiliza ambos os m�todos DI e FTIR, procurando proporcionar a melhor poss�vel detec��o de toques e marcadores fiduciais. O problema desta abordagem � que ela recebe os pontos fracos da ilumina��o DI em rela��o a ter que diminuir a interfer�ncia do fundo na detec��o e ter que controlar o ambiente em que a mesa se encontra, mas em rela��o a uma mesa exclusivamente DI os toques s�o mais bem iluminados, facilitando as suas detec��es.

\begin{figure}[h!]
	\begin{center}
	\includegraphics[scale=0.17]{virttable.jpg}
	\caption{Projeto da mesa \textit{Virttable}~[\citenum{virttable}]}
	\label{virttable}
	\end{center}
\end{figure}

\section{Softwares interativos para mesas com superf�cies multi-toque}
\label{cap2.2}

Existem v\'arios jogos produzidos para utiliza\c{c}\~ao em mesas com superf\'icies multi-toque. Por exemplo, o \textit{IRTaktiks}~[\citenum{IRTaktiks}], Figura~\ref{irtaktiks}, um jogo de estrat�gia desenvolvido para uma mesa \textit{FTIR} constru�da durante o mesmo projeto. Este jogo � para dois jogadores, onde cada jogador tem suas unidades e deve derrotar as unidades inimigas. Por�m, um problema dele � a identifica��o do jogador, pois ele n�o verifica a qual jogador pertence os toques feitos sobre a mesa, o que permite aos jogadores acidentalmente, ou n�o, controlar as unidades do inimigo, e o jogo tamb�m limita o n�mero de usu�rios para dois jogadores, algo que deve ser evitado em mesas multi-toque. 

\begin{figure}[h!]
	\begin{center}
	\includegraphics[scale=0.8]{irtaktiks.png}
	\caption{Jogo para mesas com superf�cie multi-toque \textit{IRTaktiks}~[\citenum{IRTaktiks}]}
	\label{irtaktiks}
	\end{center}
\end{figure}

Por se tratarem de superf\'icies normalmente sem limita\c{c}\~oes de n\'umero de toques, os jogos desenvolvidos para mesas devem focar em interatividade de v\'arios jogadores. Softwares como o \textit{The Laundry Game}~[\citenum{laundrygame}], Figura~\ref{laundry}, que \'e um jogo de separa\c{c}\~ao de roupas em categorias, e o \textit{Game of Life}, Figura~\ref{life}, feito pelo \textit{Verve Project}~[\citenum{gameoflife}], que \'e o tradicional jogo da vida onde uma bact\'eria vive dependendo do n\'umero de bact\'erias ao seu redor, s\~ao simples e divertidos, e permitem que qualquer n�mero de usu�rios participe, mas como � muito limitado o que se pode ser feito nesses softwares, normalmente os usu�rios param de os utilizar ap�s pouco tempo.

\begin{figure}[h!]
	\begin{center}
	\includegraphics[scale=0.8]{laundrygame.png}
	\caption{Jogo para mesas com superf�cie multi-toque \textit{The Laundry Game}~[\citenum{laundrygame}]}
	\label{laundry}
	\end{center}
\end{figure}

\begin{figure}[h!]
	\begin{center}
	\includegraphics[scale=0.8]{gameoflife.png}
	\caption{Software para mesas com superf�cie multi-toque \textit{Game of Life}~[\citenum{gameoflife}]}
	\label{life}
	\end{center}
\end{figure}

Outros softwares para a mesa s\~ao mais completos, possuindo dificuldade e sistema de pontua\c{c}\~ao. Um desses jogos seria o \textit{Eco Defense}~[\citenum{pedrosaulo}], Figura~\ref{ecodefense}, onde se deve destruir nuvens de polui\c{c}\~ao antes que afetem a floresta ao redor, podendo se colocar torres de defesa para ajudar, e a dificuldade sobe quanto mais se joga, aumentando a velocidade das nuvens, at\'e os jogadores perderem. Outro exemplo seria o \textit{Station Defender}~[\citenum{multphys}], Figura~\ref{defender}, onde os jogadores devem defender sua base construindo paredes ao redor dela. Ambos estes jogos proporcionam durabilidade pela dificuldade que escala e focam na sobreviv\^encia do jogador pelo maior tempo poss\'ivel.

\begin{figure}[h!]
	\begin{center}
	\includegraphics[scale=0.60]{ecodefense.jpg}
	\caption{Jogo para mesas com superf�cie multi-toque \textit{Ecodefense}~[\citenum{pedrosaulo}]}
	\label{ecodefense}
	\end{center}
\end{figure}

\begin{figure}[h!]
	\begin{center}
	\includegraphics[scale=0.60]{stationdefender.png}
	\caption{Jogo para mesas com superf�cie multi-toque \textit{Station Defender}~[\citenum{multphys}]}
	\label{defender}
	\end{center}
\end{figure}

\section{Softwares simuladores de f\'isica}
\label{cap2.3}
Um software simulador de f�sica consiste de um programa em que se possui um mundo aberto e o usu�rio pode criar o que desejar nele, com essas cria��es recebendo propriedades de um objeto tradicional e interagindo com o mundo de acordo com as leis da f�sica.

Um jogo que abriu as portas para a populariza��o desse tipo de software se chama \textit{Little Big Planet}~[\citenum{lbp}]. Nesse jogo, inicialmente desenvolvido para \textit{Playstation 3}~[\citenum{ps3}] e depois para \textit{Playstation Portable}~[\citenum{psp}], o jogador controla um boneco chamado \textit{Sack Boy}, que ele pode mudar a apar�ncia como quiser, e deve seguir por uma fase no estilo de jogo de plataforma at� chegar ao seu final. A diferen�a est� no estilo do jogo, que os criadores chamaram de \textit{Play, Create, Share}, onde os jogadores s�o livres para criar suas pr�prias fases como quiserem e compartilharem elas com todos os outros jogadores do mundo. Neste modo, o \textit{Sack Boy} pode colocar objetos em jogo para construir a fase, e os objetos se interagem de acordo com as leis da f�sica, podendo possuir movimento e atrapalhar ou ajudar um ao outro, sendo perigosos ou n�o para o \textit{Sack Boy}. Com esta ferramenta, os jogadores constru�ram in�meras fases inovadoras ao redor do mundo, inventando coisas que at� os criadores nunca haviam imaginado. O sucesso foi tanto que os criadores do \textit{Little Big Planet} j� lan�aram recentemente a sua continua��o, \textit{Little Big Planet 2}~[\citenum{lbp2}], onde a cria��o de fases vai al�m de somente jogos de plataforma, com o jogador podendo criar regras no jogo que permitem a cria��o de v�rios estilos de jogos diferentes, como por exemplo \textit{Shooter} ou \textit{RTS}.

\begin{figure}[h!]
	\begin{center}
	\includegraphics[scale=0.30]{lbp.jpg}
	\caption{Modo de cria��o do jogo para \textit{PS3} \textit{Little Big Planet}~[\citenum{lbp}]}
	\label{lbp}
	\end{center}
\end{figure}

Ainda no mesmo estilo de \textit{Play, Create, Share}, a mesma empresa criadora do \textit{Little Big Planet} lan�ou um jogo chamado \textit{Modnation Racer}~[\citenum{mdr}], que se trata de um jogo de corrida onde os jogadores podem criar as suas pr�prias pistas de corridas e compartilhar com todos. Lan�ado em Maio de 2010, j� possui milhares de pistas criadas por usu�rios. Esses jogos demonstram a popularidade do estilo de se criar seu pr�prio jogo.

No cen�rio de softwares simuladores de f�sica bidimensionais, tem-se um software chamado \textit{Phun}~[\citenum{phun}], Figura~[\ref{phun}], que n�o possui objetivo. Ele � uma ferramenta simuladora de f�sica bidimensional que permite ao usu�rio construir o ambiente que desejar e fazer o que quiser com ele, podendo desenhar formas normais ou abstratas, mudar a taxa de gravidade e fric��o dos objetos, fazer l�quido e controlar densidade dos elementos, entre outros. O jogador � livre para montar o mundo que desejar e fazer o que quiser com ele, sem possuir qualquer objetivo.

\begin{figure}[h!]
	\begin{center}
	\includegraphics[scale=0.40]{Phun.jpg}
	\caption{Jogo para PC \textit{Phun}~[\citenum{phun}]}
	\label{phun}
	\end{center}
\end{figure}

Um jogo muito popular neste estilo � o \textit{Crayon Physics}~[\citenum{crayonphysics}]. Neste jogo, o jogador possui um giz de cera virtual e deve-se com ele criar um cen�rio que permita que a bola que aparece em todos n�veis alcance a estrela. O jogador pode desenhar a forma que quiser, e ela � adicionada ao mundo assim que ele acabar de desenh�-la. Uma vez no mundo, o objeto obedece a gravidade e adquire um momento, obtendo assim um movimento. O jogador ainda pode utilizar pontos de fixa��o para prender objetos juntos e correntes para fazer dispositivos improvisados como elevadores. Tudo isto � feito levando em conta leis da f�sica como as tr�s leis de \textit{Newton} aplicado a um mundo bidimensional. Esse jogo foi adaptado para dispositivos multi-toque, sendo desenvolvido em Action Script por um grupo chamado \textit{Multitouch Barcelona}~[\citenum{barcelonamultcrayon}], mas contr�rio a sua inspira��o original, ele possui somente o modo de cria��o emprestado do \textit{Crayon Physics}, n�o tendo objetivo e por isso n�o podendo ser chamado de jogo.

\begin{figure}[h!]
	\begin{center}
	\includegraphics[scale=0.30]{crayon.jpg}
	\caption{Jogo para PC \textit{Crayon Physics}~[\citenum{crayonphysics}]}
	\label{crayon}
	\end{center}
\end{figure}

Inspirado no \textit{Crayon Physics}, existe outro jogo chamado \textit{Numpty Physics}~[\citenum{numptyphysics}]. Tem-se o mesmo objetivo e ferramentas neste jogo, por�m o jogo possui c�digo aberto. Este jogo foi adaptado para funcionar em mesas com superf�cies multi-toque atrav�s de um trabalho de gradua��o da Universidade T�cnica de Viena~[\citenum{multnump}], Como visto na Figura~[\ref{numpty}]. O jogo possui as mesmas fases que existiam no Numpty Physics, com algumas novas possuindo mais de uma bola para proporcionar um pouco de divers�o para um segundo jogador. Infelizmente, esse jogo � muito inst�vel e s� planeja a at� no m�ximo dois jogadores, n�o proporcionando ferramentas mais elaboradas ou desafios melhores.

\begin{figure}[h!]
	\begin{center}
	\includegraphics[scale=0.6]{numpty.png}
	\caption{Todos seis n�veis para mais de um jogador do jogo \textit{Numpty Physics} convertido para mesas com superf�cie multi-toque~[\citenum{multnump}]}
	\label{numpty}
	\end{center}
\end{figure}


\chapter{Fundamenta��o Te�rica}
\label{cap3}

Tradicionalmente, existe o costume de desenvolver jogos focados em um �nico usu�rio, algumas vezes pensando em dois ou mais, cada um entrando com seu \textit{input} via teclado ou mouse. Por�m, jogos desenvolvidos para superf�cies multitoque tem um paradigma diferente, portanto, � fundamental desenvolver suporte de entrada para um n�mero n�o determinado de usu�rios que podem participar simultaneamente com seu \textit{input} compartilhado atrav�s de uma �nica superf�cie multi-toque. Por isso s�o utilizadas t�cnicas diferentes para reconhecimento de toque e processamento de imagem para tradu��o do \textit{input}.

Neste cap�tulo, s�o apresentadas as t�cnicas para detec��o de toques. Ou seja, � detalhado como as imagens obtidas pela c�mera s�o processadas para detec��o da forma��o e movimenta��o de \textit{blobs} que correspondem aos toques e gestos sobre a mesa. Este processamento envolve a aplica��o de filtros, al�m da corre��o da imagem para eliminar distor��o. Tamb�m s�o discutidos aspectos gerais de mesas multi-toque, como o tamanho de uma mesa e a interface com o usu�rio, e por fim � analisado como trabalhar com um n�mero indeterminado de usu�rios e a detec��o de multiplos gestos sobre a mesa.

\section{Processamento de Toques}
\label{cap3.1}

Para o reconhecimento de toques � fundamental a implementa��o de um sistema de vis�o computacional. Como � explicado em~[\citenum{pedrosaulo}], a vis�o computacional trata da capacidade de computadores de capturar informa��es sobre o ambiente e process�-las, interpretando o que foi capturado. Assim, se faz necess�rio um sistema capaz de interpretar o toque em uma mesa, reconhecendo o local dos toques e extraindo das imagens obtidas informa��es sobre ele. Esse sistema � composto por um computador tradicional, uma \textit{webcam} sem filtro infravermelho e com um filtro para luz vis�vel e um projetor multim�dia. O projetor projeta a imagem em uma tela de proje��o no acr�lico da mesa, a \textit{webcam} detecta o local dos toques e os interpreta no programa que estiver executando.

\subsection{Processamento das Imagens}
\label{cap3.1.1}

As imagens ter�o de ser capturadas rapidamente em sequ�ncia, para se detectar movimento. Elas ser�o capturadas atrav�s de uma cam�ra sem filtro infravermelho e com um filtro para luz vis�vel, como um negativo de filme, de forma a simplificar sua forma e apresentar imagens mais f�ceis de serem tratadas, que n�o se confundem o toque com o fundo. Ap�s serem capturadas, as imagens ser�o filtradas de forma a destacar mais nitidamente os toques nela, e finalmente os conjuntos de p�xeis agrupados com intensidade semelhantes s�o detectados nela e mapeados, s�o os chamados \textit{blobs}, que demarcam a posi��o exata onde o toque foi realizado.

Conforme mencionado no cap�tulo anterior, existem dois m�todos principais de iluminar uma mesa multi-toque, \textit{FTIR}(\textit{Frustrated Total Internal Refraction}) e \textit{DI}(\textit{Diffuse Illumination}). \textit{FTIR} trata de uma ilumina��o em que \textit{LEDs} s�o postos em paralelo a superf�cie de acr�lico da mesa ou dentro do acr�lico e os feixes de luz ficam presos dentro do acr�lico sendo refletidos internamente devido a fen�menos de reflex�o total. Uma vez que um toque � feito no topo do acr�lico, a ilumina��o � refletida para fora da superf�cie iluminando o ponto de toque. A \textit{DI}, por outro lado, ilumina por baixo do acr�lico e normalmente enxerga tudo acima do acr�lico, incluindo o fundo al�m do ponto em que se toca. Como � apresentado em~[\citenum{virttable}], a ilumina��o por \textit{FTIR} � mais adequada para detec��o de toques, enquanto que a ilumina��o por \textit{DI} � mais indicada para se detectar a utiliza��o de marcadores fiduciais. A Figura~[\ref{iluminacoes}] demonstra as imagens capturadas por uma c�mera localizada dentro de uma mesa multitoque  para diferentes tipos de ilumina��o.

\begin{figure}[h!]
	\centering
	\mbox{\subfigure[Sem ilumina��o]{\includegraphics[scale=0.32]{virttableillumination-noillumination.png}}
	\quad
	\subfigure[FTIR]{\includegraphics[scale=0.32]{virttableillumination-FTIRonly.png}}
	\subfigure[DI]{\includegraphics[scale=0.32]{virttableillumination-DIonly.png}}
	\subfigure[FTIR e DI]{\includegraphics[scale=0.32]{virttableillumination-FTIRandDI.png}}}
	\caption{Imagens capturadas com ilumina��es diferentes~[\citenum{virttable}].}
	\label{iluminacoes}
\end{figure}

Dependendo do m�todo de ilumina��o, v�rios filtros diferentes devem ser aplicados, como explicado em~[\citenum{muller}]. A biblioteca \textit{Touchlib}, utilizada neste trabalho, possui v�rios filtros, que est�o detalhados, bem como sua utiliza��o, em~[\ref{apendiceA}].


% Na ilumina��o h�brida, deve-se aplicar todos os filtros correspondentes as duas ilumina��es para obter a melhor imagem poss�vel. Assim, os filtros a serem aplicados s�o apresentados na Figura~[\ref{filtros}]. S�o eles:
%
%
%\begin{figure}[h!]
%	\begin{center}
%	\includegraphics[scale=0.7]{filtros.jpg}
%	\caption{Imagem capturada e suas filtragens~[\citenum{pedrosaulo}].}
%	\label{filtros}
%	\end{center}
%\end{figure} 
%
%\begin{enumerate}
%	\item Imagem capturada do dispositivo (Figura~[\ref{filtros}](a)). Devido a limita��es da biblioteca \textit{Touchlib}, utilizada neste trabalho, ela deve ser composta de um canal de 8-bits, sendo convertida se necess�rio.
%	\item Filtro de \textit{Background} (Figura~[\ref{filtros}](b)), que remove o fundo da imagem, ou seja, aquilo que foi captado e que n�o � o toque do usu�rio ou marcador fiducial e est� a mais na imagem. Para tal, � armazenado um \textit{frame} na mesa sem nada em cima, que ela usa para subtrair dos pr�ximos \textit{frames} esse \textit{frame}, anulando tudo que n�o for modificado da imagem.
%	\item Filtro de \textit{Highpass} (Figura~[\ref{filtros}](c)), que aumenta o brilho de pontos a uma certa altura da mesa. Esse valor � ajustado manualmente pois depende da mesa utilizada, com o prop�sito de atenuar os toques e marcadores fiduciais.
%	\item Filtro \textit{Scaler} (Figura~[\ref{filtros}](d)), que aumenta o brilho da imagem total que restou, clareando mais os toques.
%	\item Filtro \textit{Rectify} (Figura~[\ref{filtros}](e)), que reduz os ru�dos da imagem, deixando os \textit{blobs} mais bem definidos e de f�cil detec��o.
%\end{enumerate}

\subsection{Detec��o de \textit{Blobs}}
\label{cap3.1.2}

Como dito na se��o anterior, os chamados \textit{blobs} s�o os resultados da detec��o dos toques na superf�cie. Para interpretar esses \textit{blobs}, existem dois processos essenciais, chamados de \textit{Blob-detection}, que interpreta o local em que o \textit{blob} foi criado nas imagens filtradas anteriores, e \textit{Blob-tracking}, que � a an�lise dos \textit{blobs} em uma sequ�ncia de imagens. Utilizando essas duas an�lises, pode-se identificar a cria��o de \textit{blobs}, verificar seus movimentos durante sua vida �til, bem como identificar a cria��o de novos \textit{blobs} no sistema.

A detec��o de \textit{blobs} deve ser realizada a cada frame. Da imagem final obtida da Figura~[\ref{filtros}] � observado os contornos dos \textit{blobs} que aparecem e s�o analisados para identificar se eles correspondem ao toque de dedos ou a marcadores fiduciais. Se for um toque, uma elipse � feita sobre ele, e com base nela, pode ser identificado posi��o, orienta��o e tamanho do \textit{blob}. Esse \textit{blob} ent�o � adicionado a uma lista para ser processado.

Tendo essa lista, o \textit{Blob-tracking} tem que ser feito para identificar a movimenta��o e cria��o de \textit{blobs}. Usando o frame anterior de refer�ncia, verifica se o \textit{blob} foi criado, removido ou atualizado. Para se fazer essa an�lise, usa-se 2 conjuntos, \textit{P} e \textit{Q}. O conjunto \textit{P} cont�m a lista de pontos que representam os \textit{blobs} do frame anterior. Um exemplo seria um conjunto \textit{P} com pontos (1,1) e (4,4), ilustrado no gr�fico da Figura~[\ref{grafP}].

\begin{figure}[h!]
	\begin{center}
	\includegraphics[scale=1]{GraficoP.jpg}
	\caption{Gr�fico de P~[\citenum{pedrosaulo}].}
	\label{grafP}
	\end{center}
\end{figure} 

O segundo conjunto, \textit{Q}, cont�m a lista de pontos que comp�e os \textit{blobs} do frame atual. Por exemplo, o conjunto \textit{Q} com pontos (1,2), (1,4) e (4,3), observado no gr�fico~[\ref{grafQ}].

\begin{figure}[h!]
	\begin{center}
	\includegraphics[scale=1]{GraficoQ.jpg}
	\caption{Gr�fico de Q~[\citenum{pedrosaulo}].}
	\label{grafQ}
	\end{center}
\end{figure} 

O n�mero de elementos de \textit{Q} � maior que o n�mero de elementos em \textit{P}, logo, pelo menos um \textit{blob} novo foi criado. Para saber qual o \textit{blob} criado e quais foram somente atualizados, deve-se analisar as dist�ncias entre os \textit{blobs}. Neste exemplo � demonstrado que o \textit{blob} \textit{p1} se moveu para \textit{q1}, o \textit{blob} \textit{p2} se moveu para \textit{q3}, e \textit{q2} � um novo \textit{blob} criado neste frame. A cada frame, esse conjunto \textit{Q} passa a ser o novo \textit{P}, um novo conjunto \textit{Q} � gerado, e o procedimento se repete.

\subsection{Corre��o da C�mera}
\label{cap3.1.3}

Quando uma c�mera captura uma imagem sobre uma superf�cie plana, essa imagem possui uma distor��o, onde os pontos mais distantes do centro da imagem ganham um pequeno �ngulo em rela��o a imagem original pois s�o exibidos de forma reduzida. Para se detectar os \textit{blobs}, � necess�rio executar alguns algoritmos a fim de corrigir isso e colocar toda a imagem em um mesmo plano. Como dito por \textit{Muller} em~[\citenum{muller}], v�rias bibliotecas de processamento de imagem e detec��o de multi-toque j� fazem isso. Na Figura~[\ref{distorcao}] � poss�vel visualizar uma imagem antes e depois da corre��o.

\begin{figure}[h!]
\centering
\mbox{\subfigure[Imagem da c�mera com distor��o.]{\includegraphics[scale=0.5]{distorcao(a).jpg}}
\quad
\subfigure[Imagem da c�mera corrigida.]{\includegraphics[scale=0.5]{distorcao(b).jpg}}}
\caption{Corre��o da distor��o focal da c�mera~[\citenum{muller}].}
\label{distorcao}
\end{figure}

Como explicado por \textit{Muller}, essa corre��o � feita realizando um mapeamento de pontos na imagem em rela��o a um ponto do espa�o. Para tal, � necess�rio considerar a dist�ncia focal, o centro da imagem, o tamanho de cada pixel e o coeficiente de distor��o radial da lente, propriedades da pr�pria c�mera, bem como o vetor de transla��o e a matriz de rota��o que analisam o espa�o em que se encontra a c�mera.

Assim, um ponto na c�mera pode ser corrigido de acordo com a f�rmula \textit{m = A[Rt]M}, onde \textit{A} � a matriz que possui os fatores da c�mera (Figura~[\ref{matrizA}]), \textit{M} � um ponto do espa�o, \textit{R} � a matriz de rota��o da c�mera e \textit{t} � o vetor de transla��o da mesma.

\begin{figure}[h!]
	\begin{center}
	\includegraphics[scale=1]{MatrizA.jpg}
	\caption{Matriz \textit{A}~[\citenum{pedrosaulo}].}
	\label{matrizA}
	\end{center}
\end{figure}

\section{Aspectos de mesas com superf�cie multi-toque}
\label{cap3.2}

As mesas com superf�cies multi-toque mudam a forma como o usu�rio interage com o computador. Devido a esse novo paradigma, � necess�rio repensar as capacidades oferecidas atrav�s delas ao usu�rio. Os \textit{softwares} atuais levam em considera��o uma pessoa sentada na frente de uma tela controlando um dispositivo apontador, o \textit{mouse}. Uma mesa multi-toque, no entanto, convida v�rias pessoas a partilharem simultaneamente sua utiliza��o. Neste contexto, observa-se que a maioria dos \textit{softwares} atuais n�o est�o aptos a serem utilizados neste tipo de estrutura de interface, pois n�o possuem suporte a toques m�ltiplos e simult�neos.

Al�m disso, para que as mesas tornem-se comum no dia-a-dia das pessoas, elas devem ter desempenho eficiente com a realidade dos usu�rios. Nesta se��o � discutido quest�es referentes ao n�mero de usu�rios, o n�mero de toques, os tipos de toque, a resolu��o e o tamanho da mesa.

\subsection{Tamanho e Resolu��o}
\label{cap3.2.1}

O tamanho de uma mesa multi-toque deve ser proporcional a sua utiliza��o. Uma mesa que foca na utiliza��o de poucos usu�rios deve ter tamanho suficiente para que um �nico usu�rio alcance todos seus lados, enquanto que uma mesa focada em uso em grupo de um n�mero indeterminado de usu�rios pode alcan�ar o tamanho que conseguir devido a limita��es de hardware, como o projetor multim�dia que as mesas utilizam para projetar a imagem na sua superf�cie ou como o �ngulo de abertura da c�mera utilizada para identificar os toques.

Idealmente, os \textit{softwares} devem funcionar de qualquer modo, sem saber o tamanho da mesa, para poderem ser utilizados em qualquer projeto similar. No entanto, mesmo sem este planejamento, existem maneiras de se tratar isso. A mesa pode ser projetada para se utilizar um dispositivo apontador, permitindo a um usu�rio que alcance pontos mais distantes, ou ent�o pode possuir uma fun��o que mude o tamanho da tela e a posicione mais perto do usu�rio.

Em geral, as mesas possuem um tamanho maior que os monitores de v�deo tradicionais, o que produz um problema sobre a resolu��o da imagem projetada. A maioria dos projetores possuem resolu��o padr�o de 800x600 ou 1024x768 pixels, o que, para mesas grandes, � pouco e produz uma imagem distorcida. Para contornar isso, a solu��o � utilizar um projetor \textit{short-throw} de alta resolu��o \textit{WXGA (Wide eXtended Graphics Array)}, capazes de atingir resolu��o de alta defini��o \textit{HD} de \textit{720p} ou at� \textit{1080p}.

\subsection{Interfaces}
\label{cap3.2.2}

Como citado por Brand�o et al.~[\citenum{pedrosaulo}], interfaces sens�veis ao toque trazem ao usu�rio a possibilidade de interagir diretamente com os dados. Isso traz um grande apelo ao usu�rio, pois al�m de ser natural, � de f�cil aprendizado. Por n�o possuir partes m�veis, � interessante para ser usado, tamb�m, em equipamentos acess�veis ao p�blico, tal como caixas autom�ticos e pontos de acesso.

Como citado no in�cio deste cap�tulo, os \textit{softwares} existentes s�o desenvolvidos para utiliza��o atrav�s de teclado e \textit{mouse}. No caso do \textit{mouse}, a �rea apontado pelo cursor � bem pequena, logo, as interfaces possuem poucos p�xeis para a sele��o. No entanto, em interfaces sens�veis ao toque h� uma configura��o diferente, pois o dedo humano � maior que o ponteiro do \textit{mouse}, podendo acarretar em perda de precis�o na detec��o do local do toque. Como explicado por Brand�o et al.~[\citenum{pedrosaulo}], a qualidade da c�mera � um fator determinante de precis�o da interface, pois como ela captura a imagem a ser processada, diversos aspectos como luz, dist�ncia focal e resolu��o alteram a imagem capturada afetando a usabilidade para o usu�rio. Isto porque em imagens muito claras n�o � dif�cil identificar corretamente os \textit{blobs}, sendo assim, a c�mera deve possuir um filtro de luz. A dist�ncia focal da c�mera deve suportar a dist�ncia entre a superf�cie da mesa e a c�mera para que a imagem n�o seja capturada sem foco. Al�m disso, a c�mera deve possuir uma boa resolu��o para capturar imagens com melhor qualidade.

Portanto aumentar a resolu��o, resulta em uma amostragem mais alta, o que ajuda no problema de precis�o da c�mera. Por�m, o problema da detec��o do tamanho do toque tem que ser considerado durante a constru��o da interface, pois o toque pode ser maior que o bot�o desejado e tem que se estimar onde realmente se deseja identificar o toque. Em geral o ponto tomado como refer�ncia da posi��o do toque � o seu ponto central.

Al�m desses problemas,  existe a quest�o do ponto de vis�o do usu�rio. Quando um usu�rio utiliza um computador, ele est� acostumado a possuir uma interface virada para ele. No caso da mesa, h� v�rios usu�rios e eles podem estar olhando a mesa de �ngulos diferentes. A interface deve ser agrad�vel para todos os usu�rios que forem utilizar a mesa. A diferen�a de �ngulo de vis�es � melhor exemplificado na Figura~[\ref{OK}].

\begin{figure}[h!]
	\begin{center}
	\includegraphics[scale=1]{ok.jpg}
	\caption{Vis�o de um bot�o tradicional \textit{OK} em duas vis�es diferentes~[\citenum{pedrosaulo}].}
	\label{OK}
	\end{center}
\end{figure}

Assim, ao se planejar um software para uma mesa multi-toque, deve-se prever a rota��o da interface de modo a acomodar a aplica��o de acordo com a posi��o do(s) usu�rio(s).

\subsection{N�mero de Usu�rios}
\label{cap3.2.3}

A maior dificuldade em se manusear um n�mero desconhecido de usu�rios � identificar a quem pertence determinado toque, bem como atribuir as propriedades necess�rias a cada um deles. Toma-se o exemplo usado em~[\citenum{pedrosaulo}], que cita o caso de uma aplica��o do desenho em que um usu�rio deseja trocar a cor de seu pincel de amarelo para azul, acarretando em um problema na decis�o de trocar a cor para somente o usu�rio desejado.

A solu��o, nesse caso, � trocar a cor para todos os toques, ou ent�o implementar a divis�o de �reas de trabalho, onde cada usu�rio possui sua respectiva �rea e modifica��es feitas em sua �rea n�o afetam as de outros usu�rios.

\subsection{Detec��o de Gestos}
\label{cap3.2.4}

A maioria dos \textit{softwares} leva em considera��o gestos usando somente o \textit{input} de um �nico \textit{mouse}. Assim, s�o implementados normalmente gestos com um ponto de contato. Tradicionalmente, existem o seguintes gestos:

\begin{itemize}
	\item{\textit{Click} (Figura~[\ref{ClickDrag}(a)])}: Gesto mais simples e tradicional. � detectado ao se estabelecer um ponto de contato que � retirado logo em seguida.
	
	\begin{figure}[h!]
	\centering
	\mbox{\subfigure[\textit{Click}]{\includegraphics[scale=0.5]{click.png}}
	\quad
	\subfigure[\textit{Drag}]{\includegraphics[scale=0.5]{drag.png}}}
	\caption{Gestos de \textit{Click} e \textit{Drag}~[\citenum{gestos}].}
	\label{ClickDrag}
	\end{figure}
		
	\item{\textit{Drag} (Figura~[\ref{ClickDrag}(b)])}: Continua��o do gesto acima (Click), identificado ao se estabelecer um ponto de contato seguido de movimenta��o antes da retirada do contato.
	
\end{itemize}

Poucos \textit{softwares} implementam alguns tipos de gestos usando somente um ponto de contato compostos de desenhos, como por exemplo no jogo \textit{Black and White} em que o jogador pode selecionar um poder diferente dependendo do movimento que faz com o mouse, como observado na Figura~[\ref{blackwhite}].

\begin{figure}[h!]
	\begin{center}
	\includegraphics[scale=1.5]{blackwhite.jpg}
	\caption{Reconhecimento de gesto no jogo \textit{Black and White}~[\citenum{blackandwhite}].}
	\label{blackwhite}
	\end{center}
\end{figure}

Para identificar esse gesto � preciso identificar a forma que o usu�rio est� desenhando na tela. Isso � feito identificando a dire��o em que o jogador est� desenhando e o �ngulo que ele est� seguindo. Como o ser humano n�o possui uma precis�o exata para desenhar, esses valores tem que ser aproximados e comparados a um banco de dados de gestos conhecidos, e aquele que se aproximar mais do padr�o, dentro de um limite, � o gesto detectado. Caso n�o se aproxime de nenhum gesto conhecido, o gesto n�o � reconhecido e o usu�rio deve entrar com outro gesto. 

Em uma mesa multi-toque a possibilidade de identifica��o de v�rios pontos de contato habilita a implementa��o de outros gestos, al�m dos mencionado acima. Como por exemplo:

\begin{itemize}
	\item{\textit{Rotate} (Figura~[\ref{rotateScale}(a)])}: Dois pontos de contato s�o inseridos sobre a superf�cie e s�o movimentados em dire��es opostas enquanto mant�m a dist�ncia entre os dois dedos constantes. O �ngulo entre os dois dedos � calculado e denomina a rota��o do objeto.
	
	\begin{figure}[h!]
	\centering
	\mbox{\subfigure[\textit{Rotate}]{\includegraphics[scale=0.5]{rotate.png}}
	\quad
	\subfigure[\textit{Scale}]{\includegraphics[scale=0.5]{scale.png}}}
	\caption{Gestos de \textit{Rotate} e \textit{Scale}~[\citenum{gestos}].}
	\label{rotateScale}
	\end{figure}
		
	\item{\textit{Scale} (Figura~[\ref{rotateScale}(b)])}: Analogo ao \textit{Rotate}, utiliza de dois pontos de contato sobre uma superf�cie, por�m neste caso mant�m-se o �ngulo constante e varia-se a dist�ncia entre os dedos. Uma diminui��o na dist�ncia significa uma diminui��o na escala, enquanto um aumento de dist�ncia significa um aumento de escala.
		
\end{itemize}


\chapter{Constru��o da Mesa Multi-toque}
\label{cap4}
Este cap�tulo apresenta o projeto, a constru��o e a montagem da mesa multi-toque concebida como plataforma de execu��o do \textit{software Deskworld} proposto neste trabalho. Para ilumina��o da superf�cie da mesa e reconhecimento dos toques utilizou-se o m�todo \textit{FTIR}.

\section{Estrutura da Mesa}
\label{cap4.1}

\begin{figure}[h!]
	\begin{center}
	\includegraphics[scale=0.6]{projetomesa.png}
	\caption{Projeto da mesa com superf�cie multi-toque constru�da neste trabalho.}
	\label{mesa}
	\end{center}
\end{figure}

\begin{figure}[h!]
	\begin{center}
	\includegraphics[scale=0.6]{mesalateral.png}
	\caption{Vis�o lateral do projeto da mesa com superf�cie multi-toque constru�da neste trabalho.}
	\label{mesalateral}
	\end{center}
\end{figure}

A estrutura da mesa � feita de alum�nio como material principal, de modo a prover estabilidade, leveza e mobilidade, sem prejudicar sua efici�ncia. Sua confec��o foi sob medida, pois cada elemento influi na sua configura��o. As dimens�es da mesa est�o inclusas na Figura~\ref{mesa} que apresenta o projeto da mesa. Vis�o de dentro e lateral do projeto da mesa est� ilustrada na Figura~\ref{mesalateral}.

Outra caracter�stica importante da mesa � a possibilidade de ser desmontada, o que facilita seu transporte. Apesar de possuir grandes medidas, uma vez desmontada pode ser facilmente carregada e armazenada, pois sua estrutura de alum�nio � leve. Al�m disso, mesmo montada, a mesa possui rodas com travas para facilitar a movimenta��o, podendo ser travada no local de uso. A mesa completa pode ser observada, externa e internamente, nas Figuras~\ref{mesaporfora} e~\ref{mesadentro}, respectivamente.

\begin{figure}[h!]
	\centering
	\mbox{\subfigure{\includegraphics[scale=0.18]{mesapaisagem.jpg}}}
	\mbox{\subfigure{\includegraphics[scale=0.1]{mesaretrato.jpg}}
	\quad
	\subfigure{\includegraphics[scale=0.05]{mesaporfora.jpg}}}
	\caption{Fotos da mesa multi-toque constru�da neste projeto.}
	\label{mesaporfora}
\end{figure}

\begin{figure}[h!]
	\begin{center}
	\includegraphics[scale=0.1]{mesapordentro.jpg}
	\caption{Vis�o interna da mesa, ilustrando seus componentes.}
	\label{mesadentro}
	\end{center}
\end{figure}

\begin{figure}[h!]
	\begin{center}
	\includegraphics[scale=0.1]{ps3eyenosuporte.jpg}
	\caption{C�mera \textit{Playstation Eye} no suporte.}
	\label{ps3eyenosuporte}
	\end{center}
\end{figure}

\section{Superf�cie}
\label{cap4.2}

A superf�cie da mesa � feita de uma chapa de acr�lico transl�cido, que oferece uma alta resist�ncia e transpar�ncia a um custo acess�vel. Seu coeficiente de refra��o � adequada e permite a ilumina��o de toda superf�cie. Sua transpar�ncia permite a capta��o dos toques com clareza pela c�mera. O acr�lico tem dimens�es de 120 x 90 cm e espessura de 1 cm. Para reter a proje��o � aplicado um papel vegetal sob a face interna do acr�lico. O objetivo desta � impedir que a luz do canh�o do projetor multim�dia reflita no acr�lico causando uma interfer�ncia detect�vel pela c�mera. Al�m disso, ele ajuda a destacar a ilumina��o dos dedos, quando comparadas a da m�o, facilitando a detec��o de toques.

\section{Ilumina��o Infravermelha}
\label{cap4.3}

Para um reconhecimento satisfat�rio dos toques, foi utilizado o m�todo de ilumina��o \textit{FTIR}, introduzido no cap�tulo anterior. Os \textit{LED}s infravermelhos foram posicionados em furos ao redor do acr�lico apontados paralelamente a sua superf�cie (Figura~\ref{furosacrilico}). Aproveitando a resist�ncia do acr�lico, furou-se a lateral do acr�lico a, aproximadamente, cada 1cm utilizando uma furadeira de bancada para o posicionamento e fixa��o dos \textit{LED}s. Foram realizados 200 furos no total, 115 em um dos lados da mesa com 120 cm e 85 em um dos lados com 90 cm. Os \textit{LED}s foram ent�o inseridos nos furos e soldados (Figura~\ref{ledsolda}). Os \textit{LED}s s�o diodos de 5 mm de di�metro com pot�ncia 1,1 w m�xima, tendo a sua tens�o m�xima 5 V e corrente 220 mA. Como � utilizada uma fonte com tens�o de 12,2 V, pot�ncia m�xima de 60 w e corrente m�xima de 5 A, os \textit{LED}s foram soldados de acordo com o diagrama da Figura~\ref{diagftir}, com s�ries de 8 \textit{LED}s e um resistor para regular a corrente, todas essas s�ries em paralelo entre si, de acordo com recomenda��o de uso dos \textit{LED}s calculada em~[\citenum{ledwizard}].

\begin{figure}[h!]
	\begin{center}
	\includegraphics[scale=0.08]{furosacrilico(b).jpg}
	\caption{Furos efetuados na borda do acr�lico.}
	\label{furosacrilico}
	\end{center}
\end{figure}

\begin{figure}[h!]
	\begin{center}
	\includegraphics[scale=0.08]{ledsolda.jpg}
	\caption{\textit{LED}s soldados na borda do acr�lico.}
	\label{ledsolda}
	\end{center}
\end{figure}

\begin{figure}[h!]
	\begin{center}
	\includegraphics[scale=1]{circuito.jpg}
	\caption{Esquema de montagem do circuito dos \textit{LED}s infravermelhos.}
	\label{diagftir}
	\end{center}
\end{figure}


\section{C�mera}
\label{cap4.4}

Para capturar a imagem a ser processada de modo a identificar os toques sobre a superf�cie, foi utilizada uma c�mera \textit{PlayStation Eye}, que pode ser observada na Figura~\ref{ps3eyenosuporte}. A escolha foi baseada no baixo custo da c�mera, seu �ngulo de abertura, pois permite a captura de imagens da superf�cie inteira com apenas uma c�mera, e a alta taxa de frames por segundo em sua faixa de pre�o. Sua resolu��o pode ser configurada em 640 x 480 \textit{pixels}, ou em 320 x 240 \textit{pixels} numa taxa de atualiza��o de 75 ou 120 frames por segundo, respectivamente.

Para capturar os movimentos sobre a mesa com maior precis�o, modificou-se a c�mera de modo a permitir a filtragem do espectro de luz vis�vel e captar apenas luz infravermelha. Isto foi feito primeiro removendo o filtro nativo que a c�mera possui, que bloqueava a ilumina��o infravermelha, e depois aplicando um filme fotogr�fico sobre a lente da c�mera, que filtra a luz vis�vel mas permite a passagem de luz infravermelha normalmente.

\section{Projetor}
\label{cap4.5}

Nesta mesa, o projetor utilizado � da marca \textit{BenQ} modelo \textit{MP772ST} (Figura~\ref{projetor}). Trata-se de um projetor \textit{WXGA}(\textit{Wide eXtended Graphics Array}) de resolu��o padr�o 1024x768 pixels com suporte a proje��o at� 1080i. O projetor foi posicionado a uma dist�ncia de 90cm da superf�cie do acr�lico. O projetor � fixado ao fundo da mesa e utiliza-se de um espelho para refletir a imagem at� o topo, sobre o papel vegetal que se encontra diretamente abaixo da chapa de acr�lico.

\begin{figure}[h!]
	\begin{center}
	\includegraphics[scale=0.4]{BENQ_MP772_ST.jpg}
	\caption{Projetor multim�dia \textit{BenQ MP772ST}~[\citenum{projpic}]}
	\label{projetor}
	\end{center}
\end{figure}

\section{Filtros de Processamento de V�deo}
\label{cap4.6}

Para detectar corretamente os toques, � necess�rio processar cada imagem recebida pela c�mera de forma a separar os toques das interfer�ncias das imagens de fundo. Para isso, utilizou-se uma cadeia de filtros dispon�veis na biblioteca \textit{Community Core Vision} (\textit{CCV})~[\citenum{ccv}].

\begin{figure}[h!]
	\centering
	\mbox{\subfigure[Captura sem filtro]{\includegraphics[scale=0.25]{filtro1.png}}
	\quad
	\subfigure[Filtro de Elimina��o do Fundo]{\includegraphics[scale=0.25]{filtro3.png}}
	\quad
	\subfigure[Filtro de Passo-alto]{\includegraphics[scale=0.25]{filtro4.png}}}
	\mbox{\subfigure[Filtro Amplifica��o]{\includegraphics[scale=0.25]{filtro5.png}}
	\quad
	\subfigure[Filtro de Retifica��o]{\includegraphics[scale=0.25]{filtro6.png}}}
	\caption{Aplica��o dos filtros � imagem capturada pela c�mera do sistema.}
	\label{filtros}
\end{figure}

A Figura~\ref{filtros}(a) mostra a captura direta da c�mera, sem nenhum filtro de processamento aplicado, apenas o filtro de hardware apresentado na Se��o~\ref{cap4.4}. Na Figura~\ref{filtros}(b), pode-se ver o filtro de elimina��o de fundo, que guarda uma imagem inicial como refer�ncia e depois subtrai as imagens subsequentes, de modo a remover grande parte da interfer�ncia do fundo e de imperfei��es do acr�lico. Em seguida, na Figura~\ref{filtros}(c), � aplicado o filtro de passo-alto, que aumenta a luminosidade dos pontos que est�o a determinada dist�ncia da c�mera, removendo a luminosidade de pontos al�m destes, de modo a reduzir ru�dos. J� na Figura~\ref{filtros}(d), tem-se o filtro amplifica��o, que intensifica a imagem obtida at� o momento, proporcionando \textit{blobs} brilhosos e de f�cil detec��o. Finalmente, na Figura~\ref{filtros}(e), � aplicado o filtro de retifica��o, que elimina pequenas interfer�ncias restantes, pois filtra \textit{blobs} abaixo de um certo n�vel de tamanho. Neste caso, � imprescind�vel que somente os toques estejam vis�veis, pois esta � a imagem usada para detect�-los.
\chapter{\textit{Software Deskworld}}
\label{cap5}

Neste cap�tulo � apresentado o projeto do \textit{Software Deskworld} (Figura~\ref{deskworldmesa}) e seus detalhes de implementa��o, tais como arquitetura e a estrutura empregada.

\begin{figure}[h!]
	\begin{center}
	\includegraphics[scale=0.1]{fotodeskworldmesa.JPG}
	\caption{\textit{Deskworld} sendo utilizado na mesa.}
	\label{deskworldmesa}
	\end{center}
\end{figure}

\section{\textit{Design} do \textit{Deskworld}}
\label{cap5.1}

Nesta se��o, apresenta-se o projeto do \textit{Software}, isto �, o projeto de sua aplica��o com as caracter�sticas expressas em uma vis�o computacional de alto n�vel, ou seja, a n�vel de usu�rio.

\subsection{Requisitos de interface}
\label{cap5.1.1}

De maneira simplificada, os requisitos de interface do \textit{Deskworld} podem ser expressos da seguinte forma:
\begin{itemize}
	\item sua mec�nica deve permitir a participa��o de diversos jogadores simultaneamente;
	\item deve suportar e processar toques simult�neos;
	\item deve possuir capacidade de aproximar linhas.
\end{itemize}

\subsection{Conceito}
\label{cap5.1.2}
O simulador de f�sica \textit{Deskworld} tem como principal proposta deixar a cargo da criatividade do(s) usu�rio(s) a constru��es de cen�rios, objetos e seus respectivos objetivos em seu mundo virtual. A meta principal � prover ferramentas aos usu�rios, auxiliando-os nessa constru��o. Assim, objetos e elementos f�sicos s�o representados em um plano bidimensional, regidos pelas leis da f�sica. � poss�vel desenhar um objeto e modificar suas propriedades tais como massa, coeficiente de atrito e elasticidade. O usu�rio pode aplicar uma for�a a este objeto como a gravidade, fix�-lo no plano, entre outras op��es. Pode-se ainda, alterar as propriedades do mundo em que se encontra e, sem interrup��es, divid�-lo, criando novos submundos para interagir separadamente.

\subsection{Regras e Objetos}
\label{cap5.1.3}

\begin{figure}[h!]
	\begin{center}
	\includegraphics[scale=0.6]{screen.png}
	\caption{Tela do \textit{software Deskworld}.}
	\label{jogo}
	\end{center}
\end{figure}

Basicamente, pode-se criar qualquer objeto bidimensional, com o aux�lio das ferramentas fornecidas. A constru��o funciona similarmente ao desenho virtual, onde o dedo representa o l�pis e a superf�cie da mesa o papel. Os objetos podem ser criados de forma livre, com o usu�rio desenhando-os como quiser, ou com aux�lio do software, atrav�s de suas ferramentas ou utilizando a forma livre que, ao fechar uma forma, tenta aproximar os desenhos do usu�rio a pol�gonos convexos, como pode ser observado na Figura~\ref{aprox}.

\begin{figure}[h!]
	\begin{center}
	\includegraphics[scale=0.6]{aproximacao.png}
	\caption{Aproxima��o de formas a pol�gonos convexos.}
	\label{aprox}
	\end{center}
\end{figure}

Para a altera��o de regras do mundo, ferramentas utilizadas ou propriedades dos objetos, o usu�rio possui um menu, que pode ser acessado atrav�s de dois toques subsequentes em um local dispon�vel da tela. Os itens do menu podem ser selecionados com um �nico toque neles.

Cada objeto pode ter as seguintes caracter�sticas:
\begin{itemize}
	\item Densidade
	\item Coeficiente de atrito
	\item Coeficiente de restitui��o de for�a
	\item For�a normal
	\item Massa
	\item Volume
	\item Velocidade
\end{itemize}

Dentre as caracter�sticas acima, a densidade, o coeficiente de atrito e o coeficiente de restitui��o de for�a podem ser alterados pelo usu�rio a qualquer momento, utilizando o menu de objetos, acessado da mesma maneira em que se acessa o menu tradicional, mas com o duplo clique sendo em cima de um objeto ao inv�s de um local dispon�vel da tela. As demais s�o calculadas automaticamente, de acordo com as intera��es dos objetos com o mundo, como por colis�es, movimenta��es ou altera��es de gravidade.

%Al�m dessas propriedades, os objetos podem interagir entre si por meio de juntas, colis�es ou motores adicionados aos objetos, que aplicam uma for�a constante sobre um objeto. As juntas s�o entidades que permite fixar um objeto a outro. Funciona, analogamente, a colocar um prego para juntar dois peda�os de madeira.
Al�m dessas propriedades, a cor dos objetos pode ser alterada pelo menu, bem como a gravidade do mundo. Tamb�m � poss�vel a separa��o de mundos com gravidades diferentes e cores de pinc�is diferentes entre si. O usu�rio ainda tem acesso a um apagador, para retirar do mundo qualquer objeto que n�o queira mais. Caso o usu�rio apague uma divis�o entre mundos, os mundos continuam divididos e com suas propriedades diferentes, mas os objetos existentes em cada mundo podem transitar livremente entre os mundos unidos. A �ltima propriedade dispon�vel no menu � o \textit{Mute}, que pode ser selecionado para se desligar a m�sica do jogo caso algum usu�rio deseje, ou ligar novamente caso esteja desligada.

As ferramentas e objetos dispon�veis est�o ilustradas na Figura~\ref{jogo}.

Neste \textit{software} interativo, n�o existem regras pr�-definidas. Fica a crit�rio do usu�rio definir suas regras enquanto interage com o \textit{software} criando cen�rios e objetos.

\subsection{Interface}
\label{cap5.1.4}
A interface utilizada para testes � a mesa cuja constru��o foi descrita no Cap�tulo~\ref{cap4}. Entretanto, salienta-se que o \textit{software Deskworld} � port�vel para qualquer interface multi-toque que suporte o protocolo \textit{TUIO}.

\section{Ferramentas de suporte}
\label{cap5.2}

Nesta se��o, ser�o descritas as ferramentas de suporte, que fornecem aplica��es reutiliz�veis para facilitar o desenvolvimento de \textit{softwares}, utilizadas na constru��o do simulador \textit{Deskworld}, tais como game engine e bibliotecas.

\subsection{\textit{Game Engine}}
\label{cap5.2.1}

A \textit{engine} utilizada no desenvolvimento deste software foi a Box2D~[\citenum{Box2Dmanual}]. Esta engine � livre, ou seja, pode ser utilizada sem nenhum custo. A principal funcionalidade da \textit{Box2D}, no contexto desse \textit{software}, foi auxiliar na cria��o dos objetos e no tratamento de eventos relacionados � simula��o das leis da f�sica. Tais eventos s�o:

\begin{itemize}
	\item Gravidade
	\item Colis�o
	\item For�a de a��o e rea��o
	\item For�a de atrito
\end{itemize}

Al�m dessas funcionalidades, a \textit{Box2D} � respons�vel pela ger�ncia de mem�ria, jun��o de objetos e/ou a fixa��o deles. Sua escolha foi baseada em boa performace, simula��o f�sica de ambientes bidimensionais, al�m da estabilidade de seu c�digo.

\subsection{\textit{Bibliotecas de apoio}}
\label{cap5.2.2}

Para facilitar a implementa��o do \textit{software}, foram utilizadas diversas bibliotecas. Uma das bibliotecas padr�o do C++, a \textit{STL} \textit{(Standart Template Library)}, possui algumas estruturas complexas de dados j� implementadas, sendo amplamente utilizada neste simulador. Para a parte gr�fica do jogo foi utilizada a biblioteca \textit{SDL}~[\citenum{sdl}]  \textit{(Simple Direct Layer)} em conjunto com a \textit{OpenGL}~[\citenum{opengl}]  (\textit{Open Graphics Library}). Com rela��o ao �udio, utilizou-se a biblitoca \textit{SDL\_mixer}~[\citenum{sdlmixer}] . Para tratamento de \textit{input} foi utilizada a biblioteca do protocolo \textit{TUIO}~[\citenum{TUIO}] (\textit{Tangible User Interface System}) para receber as mensagens enviadas pelo \textit{CCV}~[\citenum{ccv}] (\textit{Community Core Vision}) explicado mais a frente.

\subsubsection{\textit{SDL} e \textit{OpenGL}}
\label{cap5.2.2.1}

Por oferecer uma boa abstra��o de \textit{hardware}, ser bem difundida no mercado de jogos e possuir ampla documenta��o, utilizou-se a \textit{SDL} para disponibilizar o acesso ao ambiente. Em conjunto com a \textit{SDL}, na parte gr�fica, foi utilizada a \textit{OpenGL}, que realiza a renderiza��o das imagens.

A \textit{SDL\_mixer} � utilizada para permitir a reprodu��o de diversas faixas de �udio simultaneamente.

\subsubsection{Protocolo \textit{TUIO}}
\label{cap5.2.2.2}

A comunica��o entre o usu�rio da mesa e o aplicativo � realizada de acordo com o protocolo \textit{TUIO}~[\citenum{TUIO}]. Este protocolo � especificado para atender as necessidades de comunica��o das interfaces tang�veis, que s�o interfaces sens�veis ao toque, capazes de serem controladas por movimentos corporais e gestos. A implementa��o � simples e visa melhorar a performance na comunica��o. Para isso, ele opera sobre a camada \textit{UDP}(\textit{User Datagram Protocol}) de transporte utilizando tr�s tipos de mensagens: \textit{set}, \textit{alive} e \textit{fseq}. As mensagens \textit{set} s�o utilizadas para informar o estado de um objeto. Mensagens \textit{alive} indicam o conjunto de objetos presentes na interface atrav�s de uma identifica��o �nica atribu�da a cada novo elemento reconhecido. Mensagens \textit{fseq} s�o transmitidas antes da etapa de atualiza��o de cada quadro, para marc�-lo unicamente, associando-o a cada mensagem \textit{set} e \textit{alive}. A seguir � apresentado um  resumo do funcionamento do protocolo:

\begin{itemize}
	\item Par�metros do objeto s�o enviados ap�s mudan�a de estado, por interm�dio da mensagem \textit{set}.
	\item Objetos removidos da interface s�o comunicados por meio de mensagens \textit{alive}.
	\item Cliente deduz a lista de objetos adicionados e removido por meio das mensagens \textit{set} e \textit{alive}.
	\item Mensagens \textit{fseq} associam um ID a um conjunto de mensagens \textit{set} e \textit{alive} do quadro.
\end{itemize}

Apesar da camada UDP de transporte n�o oferecer garantia de entrega dos dados, o TUIO implementa redund�ncia de informa��o para se precaver contra a perda de dados na transmiss�o. Al�m disso, o estado de um objeto, mesmo inalterado, � enviado periodicamente em uma mensagem \textit{set}. Portanto, o protocolo � adequado para ser utilizado em aplicativos controlados por interfaces multitoque, otimizando a intera��o e confiabilidade da comunica��o. Na Tabela~\ref{tab:tabela1_param_TUIO}, pode-se observar todos os par�metros que s�o enviados a cada mensagem do protocolo \textit{TUIO}. A posi��o, tal como vetor de movimento, acelera��o, largura e altura do \textit{blob} s�o vari�veis do tipo flutuante, pois o protocolo TUIO mapeia a tela a n�vel com valores entre 0 e 1.

\begin{table}[htbp]
  \centering
  	\caption{Par�metros de mensagens TUIO em superf�cies interativas 2D}
    \begin{tabular}{|l|l|l|} %rrr}
    \textbf{Par�metros} & \textbf{Significado dos par�metros} & \textbf{Tipo} \\
    s     & sessionID (ID tempor�rio do objeto) & int32 \\
    x, y  & posi��o & float32 \\
    X, Y  & vetor de movimento (velocidade de movimento e dire��o) & float32 \\
    m     & acelera��o de movimento & float32 \\
    w, h  & largura e altura do \textit{blob} & float32 \\
    \end{tabular}%
  \label{tab:tabela1_param_TUIO}%
\end{table}%

\subsubsection{\textit{Community Core Vision}}
\label{cap5.2.3}

\begin{figure}[h!]
	\begin{center}
	\includegraphics[scale=0.4]{ccv.jpg}
	\caption{Aplicativo de captura de toques \textit{Community Core Vision}~[\citenum{ccv}]}
	\label{ccv}
	\end{center}
\end{figure}

O \textit{CCV} (\textit{Community Core Vision})~[\citenum{ccv}] � um aplicativo para auxiliar no processamento de imagens capturadas pela c�mera em intera��es com superf�cies multi-toque. A Figura~\ref{ccv} apresenta um \textit{screenshot} do aplicativo. Ele lida com o acompanhamento dos \textit{blobs} de luz infravermelha, enviando mensagens para o \textit{Deskworld}, tais como o encostar do dedo, o deslocar do dedo e o retirar do dedo. Interage com a maioria dos dispositivos de captura de v�deo, sendo muito �til para um projeto de software que utilize uma mesa multitoque.  Atualmente, possui suporte para plataformas \textit{Windows} e \textit{Linux}, 32 ou 64 \textit{bits}, e \textit{MacOS}. Ele permite, por padr�o, uma quantia m�xima de 20 toques simult�neos, por�m este n�mero pode ser modificado em seu arquivo de configura��o. Neste trabalho, foram testados at� 30 toques ao mesmo tempo diretamente com a mesa, por�m o valor de toques m�ximos possui a possibilidade de ser incrementado ainda mais.

Seu funcionamento consiste na aplica��o de filtros configur�veis, como os mostrados na Figura~\ref{filtros} no cap�tulo anterior. Em seu arquivo de configura��o, \textit{config.xml}, pode-se modificar os dados da c�mera, tais como resolu��o e frames por segundo; o \textit{CCV} ir� automaticamente ajustar a c�mera selecionada para os dados mais pr�ximos suportados. A execu��o do \textit{CCV} funciona de acordo com o paradigma cliente-servidor, onde o mesmo atua como servidor e a aplica��o como cliente. Esta comunica��o cliente-servidor � feita utilizando o protocolo TUIO~[\citenum{TUIO}] \textit{(Tangible User Interface System)}, utilizando pontos normais x e y enviados via \textit{TCP}, ou em pacotes especiais para aplicativos \textit{Flash}, permitindo uma arquitetura distribu�da, onde o aplicativo � executado separadamente da aplica��o de processamento de imagens.

\section{Arquitetura e Detalhes de Implementa��o}
\label{cap5.3}

A evolu��o da tecnologia dos \textit{hardwares} para computadores pessoais proporcionou grande desenvolvimento � ind�stria de jogos eletr�nicos, levando-a a aprimorar seu produto ao longo do tempo. Este caminho foi marcado pela forma��o de grandes equipes e pela extensiva carga de codifica��o. Os jogos eletr�nicos mais modernos podem ultrapassar um milh�o de linhas de c�digo de acordo com \textit{Rabin et al.}~[\citenum{I2GMDVP}]. Portanto, foi necess�rio organizar estes extensivos c�digos a fim de obter um melhor desempenho no desenvolvimento de novos jogos. Esta organiza��o pode ser realizada por meio da ado��o de uma arquitetura bem definida.

\subsection{Arquitetura do \textit{Deskworld}}
\label{cap5.3.1}

Apesar de n�o se encaixar na defini��o de jogo eletr�nico, o \textit{Deskworld} possui muitas caracter�sticas semelhantes, pois � um \textit{software} interativo, onde o usu�rio se relaciona com objetos em um mundo virtual. Dessa maneira, este \textit{software} foi desenvolvido a partir da utiliza��o de um padr�o de desenvolvimento voltado a jogos. Para este \textit{software} foi escolhida uma arquitetura modular orientada � objetos.

\subsubsection{Vantagens da abordagem orientada � objetos}
\label{cap5.3.1.1}

De acordo com \textit{Rabin et al.}~[\citenum{I2GMDVP}] a maioria dos jogos se desenvolvem em torno de objetos ou entidades em um mundo virtual, nos quais o usu�rio deve realizar a��es. Antes da exist�ncia do paradigma da orienta��o a objetos, os jogos eram desenvolvidos com programa��o procedural e assim, a �nfase da programa��o baseava-se no c�digo propriamente dito. Isto porque o jogo era observado, conceitualmente, como uma sequ�ncia de c�digo com execu��o de procedimentos e fun��es. Na programa��o orientada � objetos, a perspectiva � outra, pois a �nfase ocorre no conceito de objeto que � uma cole��o de informa��es conjuntas a uma s�rie de opera��es para processar estes dados. Portanto, utilizando orienta��o � objetos � poss�vel obter \textit{softwares} que se adequam ao formato dos jogos, pois as entidades desses jogos podem ser expressas pelas classes, os objetos por inst�ncias da classe e a��es entre eles pelos m�todos.

Outra vantagem encontrada nesta abordagem � a heran�a de classes, que permite a reutiliza��o de c�digo em m�ltiplas classes que possuam parentesco. Com heran�a, � poss�vel estender caracter�sticas das superclasses para as subclasses. No \textit{Deskworld}, utilizou-se heran�a para, principalmente, padronizar e reutilizar parte do c�digo dos \textit{Game Objects}. Tamb�m importante, o suporte ao polimorfismo � crucial no desenvolvimento de jogos j� que permite a realiza��o de \textit{update} e \textit{render} de diversos componentes do \textit{software} sem precisar separ�-los em chamadas de execu��es diferentes, pois todos ser�o reconhecidos pelo seu tipo e ter�o seus respectivos m�todos executados corretamente.

A arquitetura do programa foi modularizada para especificar claramente os  seus subsistemas, facilitando a manuten��o e o entendimento do c�digo. No diagrama apresentado na Figura~\ref{diag-deskworld} pode ser observada a composi��o de m�dulos do \textit{Deskworld} e suas principais classes. A decomposi��o em m�dulos facilita a visualiza��o da implementa��o, j� que o \textit{software} possui subsistemas bem claramente definidos como os de �udio, v�deo, \textit{engine} e de \textit{input}.

\subsection{Detalhes de implementa��o}
\label{cap5.3.2}

Esta se��o apresenta a l�gica de implementa��o do \textit{software}. Primeiramente, � exposto uma vis�o geral do funcionamento do sistema. Em seguida, os subsistemas s�o detalhados e seus pap�is tra�ados.

\subsubsection{Vis�o Geral do \textit{Software}}
\label{cap5.3.2.1}

Como observado no diagrama da Figura~\ref{diag-deskworld}, existe uma classe \textit{Game Manager} respons�vel pelo gerenciamento da execu��o do programa. Suas principais a��es s�o inicializar os subsistemas de �udio, v�deo, carregar o cen�rio e inicializar a \textit{Engine} do \textit{Deskworld}. A classe �udio � respons�vel por carregar arquivos de �udio, bem como manipul�-los, podendo tocar, parar ou resumir um arquivo carregado. A \textit{Graphics} implementa solu��es para renderizar os objetos na tela e gerenciar a mem�ria de v�deo. O \textit{InputManager} trata as mensagens de entrada recebidas, explorados na Se��o~\ref{cap5.2.2.2}, e gera eventos relativos aos toques. A \textit{Engine} tem como objetivo criar, alterar e posicionar objetos no mundo, al�m de se comunicar de posse destas informa��es  com a \textit{Box2D}. A classe \textit{LevelState} concentra o maior fluxo de informa��es, pois todas as inst�ncias de classe de objetos do mundo s�o criada nela. Os m�todos de \textit{Update}, muito importante em simula��es f�sicas, e \textit{Render} t�m suas chamadas realizadas pelo \textit{LevelState}. O \textit{Game Object} armazena informa��es sobre os objetos como sua dimens�o, posi��o e estado.

\begin{figure}[h!]
	\begin{center}
	\includegraphics[scale=0.666]{diagdeclasses.jpg}
	\caption{Diagrama de classes do \textit{Deskworld}}
	\label{diag-deskworld}
	\end{center}
\end{figure}

\subsubsection{Subsistemas de �udio e V�deo}
\label{cap5.3.2.2}

O subsistema de �udio possui atributos para identificar o nome do arquivo de �udio, o tipo do arquivo e um ponteiro para seu endere�amento na mem�ria. Seu construtor carrega o arquivo de �udio para mem�ria e retorna esse endere�o ao ponteiro da classe. Utiliza-se esta classe para tocar e pausar a m�sica de fundo. 

A classe \textit{Graphics} gerencia a mem�ria de v�deo com a \textit{OpenGL}, e a utiliza para desenhar formas geom�tricas na tela. A classe \textit{ImageLoader} carrega imagens e texturas e renderiza-as. As texturas presentes no \textit{Software} s�o o fundo, menu e seus componentes.

\textbf{\textit{Singleton}}

A classe \textit{Graphics} � um \textit{Singleton}, isto �, possui, obrigatoriamente, apenas uma inst�ncia durante todo o programa. \textit{Singleton} � um \textit{design pattern}~[\citenum{gamma}] muito utilizado em situa��es que necessitam de grande controle sobre uma classe para que seja instanciada somente uma vez. Esta t�cnica consiste em declarar um construtor privado que ser� executado uma �nica vez. Constr�i-se um m�todo que retorna a inst�ncia �nica por meio de um ponteiro est�tico. Caso seja a primeira vez que � chamado, a classe � instanciada, caso contr�rio, retorna a inst�ncia j� existente. Al�m desta, o \textit{InputManager} e a \textit{Engine} s�o singletons. Isto acontece porque todas essas classes s� devem possuir somente uma inst�ncia, pois n�o se deseja v�rias inst�ncias controlando a sa�da de v�deo, a entrada de dados ou a simula��o de objetos.

\subsubsection{Subsistema de \textit{Input}}
\label{cap5.3.2.3}

Neste subsistema, para dar suporte ao protocolo TUIO, tem-se a classe \textit{InputManager}, derivada da classe \textit{TuioListener}, que implementa m�todos para escutar mensagens relativas � detec��o de toques, gerando eventos. Estes eventos podem adicionar e remover toques ou atualizar suas posi��es. Tamb�m suporta a detec��o de eventos de input via mouse ou teclado. O m�todo principal da classe � o \textit{Update}, onde s�o percorridos todos os eventos enviados durante um \textit{Game Loop}, cada um sendo analisado e seus dados separados para utiliza��o das outras classes. Por exemplo, caso seja adicionado um toque, o n�mero identificador deste toque � salvo, permitindo que outra classe possa utiliz�-lo por meio do m�todo \textit{IsTouching}, de modo a descobrir se o toque com o identificador desejado est� tocando ou n�o a superf�cie.

\subsubsection{Subsistema da \textit{Engine}}
\label{cap5.3.2.4}

A \textit{Engine} � composta pela classe \textit{Engine}, al�m do pacotes de classes da biblioteca \textit{Box2D}. Por interm�dio destas classes, ocorre a cria��o dos objetos no mundo virtual. A simula��o de gravidade, da for�a de atrito, e das for�as de a��o e rea��o, s�o calculadas pela \textit{Engine}. Um importante recurso para uma boa simula��o � a detec��o de colis�o, tratada por esse subsistema. Tamb�m s�o realizadas opera��es de destrui��o de objetos, al�m da localiza��o dos objetos ser atualizada e repassada aos outros m�dulos pela \textit{Engine}.

\subsubsection{Subsistema de ger�ncia de objetos}
\label{cap5.3.2.5}

A classe \textit{LevelState} � respons�vel pelo �nico estado do \textit{software}. Estar no estado de \textit{LevelState} significa que o \textit{software} est� em execu��o e o usu�rio pode interagir como desejar, criando objetos ou mudando propriedades. Nesta classe, todas as imagens que aparecem dentro do jogo s�o carregadas e renderizadas na tela. Seu m�todo de \textit{Update}, chamado uma vez pelo \textit{Game Loop}, verifica todos os toques existentes ou removidos e decide o que cada um ir� fazer, como por exemplo, criar um objeto novo, arrastar um objeto existente, abrir o menu ou alterar as op��es. Seu outro m�todo, \textit{Render}, � respons�vel por montar a tela do jogo, renderizando na tela todas as imagens em sua sequ�ncia, deixando-as prontas para serem mostradas na tela, assim que tudo estiver em posi��o.

A \textit{GameObject} � uma superclasse que agrega subclasses referentes a todos objetos presentes \textit{software}. Estas classes representam os objetos desenhados no programa que podem ser pol�gonos, c�rculos ou de forma livre. Al�m disso, objetos podem ser motores e juntas, explicadas no in�cio deste cap�tulo. O pr�prio mundo do software � um objeto, pois podem existir v�rios mundos, e as pr�prias barreiras usadas para dividir os submundos tamb�m s�o consideradas objetos. Todos objetos possuem v�rtices que indicam suas posi��es e suas �reas. Todos os \textit{GameObject}s possuem m�todos de \textit{Update} e \textit{Render}.
%\chapter{Arquitetura e Detalhes de Implementa��o}
\label{cap6}

\section{Arquitetura do jogo}
\label{cap6.1}

A evolu��o da tecnologia dos hardwares dos computadores pessoais proporcionou grande desenvolvimento � ind�stria de jogos eletr�nicos levando-a a aprimorar seu produto ao longo do tempo. Este caminho foi marcado por grandes equipes e pela extensiva codifica��o. Os jogos eletr�nicos mais modernos podem ultrapassar um milh�o de linhas de c�digo de acordo com \textit{Rabin et al.}~[\citenum{I2GMDVP}]. Portanto, foi necess�rio organizar estes extensivos c�digos a fim de obter melhor performace no desenvolvimento de novos jogos. Esta organiza��o pode ser realizada por meio da ado��o de uma arquitetura bem definida.

Para este jogo escolhemos uma arquitetura de desenvolvimento baseada em um sistema de componentes.

\subsection{Sistema de Componentes}
\label{cap6.1.1}

Um sistema de componentes � caracterizado pela sua estrutura n�o-hier�rquica. O jogo que se baseia em componentes ter� somente uma classe pai que representar� todas as entidades do jogo, enquanto quaisquer outras classes ser�o irm�s entre si, pois todas ser�o filhas da entidade do jogo. Cada classe filha da entidade � um componente, ou seja, o entidade do jogo � formada por uma composi��o de componentes.

� observado a independ�ncia de cada componente, onde podemos, por exemplo, ter a liberdade de transformar um elemento de uma classe jogador em ve�culo. Entretanto, mesmo com a mudan�a podemos manter a coer�ncia l�gica do jogo, pois estas classes ter�o a mesma estrutura que forma a entidade do jogo, como audio, corpo f�sico e apar�ncia.

\begin{figure}[h!]
	\begin{center}
	\includegraphics[scale=0.8]{componente.png}
	\caption{Exemplo de uso de componentes em entidades de jogo~[\citenum{pedrosaulo}]}
	\label{componente}
	\end{center}
\end{figure}

\subsection{Vantagens em rela��o ao sistema hier�rquico}
\label{cap6.1.2}

Um sistema baseado em componentes � uma alternativa ao forte acomplamento provocado pela hierarquia de classes. Em linguagens orientadas � objeto a hierarquia de classes pode ser de grande ajuda como pode dificultar a codifica��o do software. Isto porque, em sistemas onde temos tipos de objetos pr�-definidos, assim como as rela��es entre eles, a hierarquia pode ser de grande ajuda. Isto acontece pelo fato da f�cil representa��o do jogo como um hierarquia. Por exemplo, em jogos de esporte, geralmente, temos caracter�sticas fortemente consolidadas. Um jogador de basquete, dificilmente, se transformar� em uma bola. A forte acopla��o entre as classes favorece a implementa��o de uma arquitetura hierarquizada.

\begin{figure}[h!]
	\begin{center}
	\includegraphics[scale=0.65]{hierarquia.png}
	\caption{Exemplo de uso de hierarquia em entidades de jogo~[\citenum{pedrosaulo}]}
	\label{hierarquia}
	\end{center}
\end{figure}

No entanto, no caso do jogo \textit{DeskWorld} teremos problemas se utilizarmos este tipo de arquitetura. Isto porque, de acordo com \textit{Rabin et al.}~[\citenum{I2GMDVP}], uma arquitetura majoritariamente baseada em hierarquia de classes possui significantes limita��es. Uma de suas principais � a baixa flexibilidade devido ao forte acomplamento. O fato de termos classes hier�rquicas gera a depend�ncia \textit{inter-classe}, j� que, parte de suas caracter�sticas s�o herdadas de uma classe, sendo que, mudan�as realizadas em caracter�sticas de uma classe refletiram sobre suas filhas.

Outro ponto apresentado em~[\citenum{I2GMDVP}], � que, em linguagens bem difundidas no desenvolvimento de jogos, como \textit{C++} e \textit{Java}, a estrutura modelada em hierarquia � est�tica, n�o permitindo a altera��o de classes em tempo de execu��o. Isto pode ser um grande entrave, pois certos jogos podem realizar diversas altera��es dr�sticas no comportamento de entidades. Podemos exemplificar esse ponto tomando um jogo em que a morte de um certo inimigo possa se transformar em um item ou dinheiro.

\subsection{Comunica��o entre componentes}
\label{cap6.1.3}

Para aproveitar a flexibilidade provida pelo sistema de componentes, necessitamos realizar a comunica��o entre componentes, visto que, ser� a respons�vel por engatilhar a transforma��o de uma entidade. � a mensagem que avisar� ao objeto que seu comportamento ir� ser alterado e associado a uma outra classe.

Esta comunica��o pode ficar a cargo da entidade do jogo, onde, por exemplo, pode realizar a transfer�ncia de um som de uma classe de �udio para uma classe ve�culo. Num sistema simples, essa comunica��o pode ser implementada por uma simples chamada de fun��o onde ponteiros s�o passados refer�nciando suas respectivas caracter�sticas.

\section{\textit{Design Patterns}}
\label{cap6.2}

\textit{Design Patterns} s�o padr�es de projeto de orienta��o a objetos que visam reconhecer padr�es recorrentes de forma a aumentar a flexibilidade e permitir o reuso do c�digo se usados de forma adequada. Isso se d� pois o seu uso aumenta a indep�ndencia dos sistemas, o que tamb�m aumenta a capacidade de evolu��o do c�digo em si.

Como foi descrito em~[\citenum{pedrosaulo}], para cada padr�o, s�o estabelecidos os seguintes aspectos:
\begin{itemize}
	\item{O \textbf{nome} usado para descrever, de maneira sucinta e precisa em no m�ximo duas palavras, o problema, a solu��o e as consequ�ncias inerentes do padr�o.}
	\item{O \textbf{problema} e o contexto adequados para a aplica��o do padr�o.}
	\item{A \textbf{solu��o} descreve os elementos que comp�em o padr�o. Cada solu��o inclui relacionamentos entre classes e a colabora��o entre estes casos existam.}
	\item{As \textbf{consequ�ncias} de se aplicar o padr�o, ou seja, os resultados e as desvantagens.}
\end{itemize}

Gra�as ao fato de jogos serem extremamente modulares, eles ganham um grande benef�cio por utilizar alguns padr�es de projeto. Abaixo est�o explicados os mais utilizados para este prop�sito.

\subsection{\textit{Singleton}}
\label{cap6.2.1}

\begin{figure}[h!]
	\begin{center}
	\includegraphics[scale=1]{singleton.png}
	\caption{Padr�o \textit{Singleton}~[\citenum{pedrosaulo}]}
	\label{singleton}
	\end{center}
\end{figure}

O padr�o \textit{Singleton} trata de um padr�o aonde uma classe necessita necessariamente ser �nica durante todo o programa, mesmo ela podendo ser utilizada em v�rias situa��es.

Atrav�s desse padr�o, a classe \textit{Singleton} cuida para que ela seja instanciada somente uma vez. Podemos verificar a implementa��o desse padr�o na figura~[\ref{singleton}], onde temos uma opera��o de classe, 'Instance()', que retorna a �nica inst�ncia, e � salvo como ponteiro est�tico dessa classe. Caso seja a primeira vez que � chamado, a classe � instaciada, caso contr�rio, retorna a inst�ncia j� existente. O construtor � privado para que a classe n�o possa ser criada por fora dela.

Em jogos, o padr�o \textit{Singleton} � muito usado para controlar eventos e gerenciar mensagens, bem como controlar recursos. Em todos esses casos, apensa uma inst�ncia pode existir e ela deve permitir acesso global a ela.

\subsection{\textit{Observer}}
\label{cap6.2.2}

\begin{figure}[h!]
	\begin{center}
	\includegraphics[scale=1]{observer.png}
	\caption{Padr�o \textit{Observer}~[\citenum{pedrosaulo}]}
	\label{observer}
	\end{center}
\end{figure}

O padr�o \textit{Observer}, visto na figura~[\ref{observer}], define um padr�o onde um objeto, chamado de \textit{subject}, mant�m uma lista de observadores e os notifica de mudan�as de estados. Quando a informa��o � passada, os observadores consultam o \textit{subject} para sincronizar os dados. O \textit{subject} n�o precisa conhecer detalhes do observador.

Utilizando a classe \textit{subject}, podemos adicionar ou remover observadores O observador, no entanto, define uma interface para objetos que notificam sobre sua altera��o para o subject. Essa � a �nica liga��o entre o observador e o \textit{subject}.

Este padr�o tamb�m � reconhecido na ind�stria de jogo. Como a arquitetura de jogos � marcada pela intera��o entre entidades distintas, o \textit{observer} deve ser usado. Assim as entidades se conhecem e aguardam receber atualiza��es de outras entidades que comp�em o seu meio ambiente.
%\chapter{Estrutura de implementa��o do simulador}
\label{cap7}

Neste cap�tulo, ser� descrito as ferramentas de suporte utilizadas na constru��o do simulador.

\section{\textit{Game Engine}}
\label{cap7.1}

A \textit{engine} utilizada neste jogo foi a \textit{Box2D}.Seu manual pode ser encontrado em~[\citenum{Box2Dmanual}]. A principal funcionalidade da \textit{Box2D} neste jogo foi auxiliar na cria��o dos objetos e no tratamento de eventos relacionados a simula��o das leis da f�sica. Estes eventos est�o apresentados abaixo:

\begin{itemize}
	\item Gravidade
	\item Colis�o
	\item For�a de a��o e rea��o
	\item For�a de atrito
	\item For�a de fric��o
\end{itemize}

Al�m dessas funcionalidades, a \textit{Box2D} � respons�vel pela ger�ncia de mem�ria, jun��o de objetos e/ou fixa��o deles.

\section{\textit{Bibliotecas de apoio}}
\label{cap7.2}

Para facilitar a implementa��o de foram utilizadas diversas bibliotecas. Uma das bibliotecas padr�o do C++ a STL \textit{(Standart Template Library)} possui algumas estruturas complexas de dados j� implementadas, sendo amplamente utilizada neste simulador. Para a parte gr�fica do jogo foi utilizada a biblioteca SDL \textit{(Simple Direct Layer)} conjuntamente com a \textit{OpenGL} (\textit{Open Graphics Library}). Com rela��o ao �udio, utilizou-se a SDL\_mixer. Para tratamento de \textit{input} foi utilizada a \textit{Touchlib}.

\subsection{SDL e \textit{OpenGL}}
\label{cap7.2.1}

Por oferecer uma boa abstra��o de \textit{hardware}, ser bem difundida no mercado de jogos e possuir ampla documenta��o, utilizou-se a SDL para disponibilizar o acesso ao ambiente \textit{OpenGL}. Conjuntamente com a SDL, na parte gr�fica, foi utilizado a \textit{OpenGL} (\textit{Open Graphics Library}) que realiza a renderiza��o das imagens.

A SDL\_mixer � utilizada para permitir a reprodu��o de diversas faixas de audio simultaneamente.

\subsection{Protocolo \textit{TUIO}}
\label{cap7.2.2}

Comunica��o entre o usu�rio da mesa com o aplicativo � realizada de acordo com o protocolo \textit{TUIO}~[\citenum{TUIO}] (\textit{Tangible User Interface System}). Este protocolo especificado para atender as necessidades da comunica��o das interfaces tang�veis. Interfaces tang�veis s�o interfaces sens�veis a toque, capazes de serem controladas por movimentos corporais e gestos. A implementa��o � simples e visa melhorar a performace na comunica��o. Para isso, ele opera sobre a camada UDP(\textit{User Datagram Protocol}) de transporte utilizando tr�s tipos de mensagens: \textit{set}, \textit{alive} e \textit{fseq}. Mensagens \textit{set} s�o utilizadas para informar o estado de um objeto. Mensagens \textit{alive} indicam o conjunto de objetos presentes na interface atrav�s de uma identifica��o �nica atribu�da a cada novo elemento reconhecido. Mensagens \textit{fseq} s�o transmitidas antes da etapa de atualiza��o de cada quadro para marcar-lo unicamente, associando-o a cada mensagem \textit{set} e \textit{alive} dele. Resumindo o funcionamento do protocolo:

\begin{itemize}
	\item Par�metros do objeto s�o enviados ap�s mudan�a de estado atrav�s da mensagem \textit{set}
	\item Objetos removidos da interface s�o comunicados atrav�s de mensagens \textit{alive}
	\item Cliente deduz a lista de objetos adicionados e removido por meio das mensagens \textit{set} e \textit{alive}
	\item Mensagens \textit{fseq} associam um ID a um conjunto de mensagens \textit{set} e \textit{alive} do quadro
\end{itemize}

Apesar da camada UDP de transporte n�o oferecer garantia de entrega dos dados, o TUIO implementa redund�ncia de informa��o para se precaver contra a perda de dados na transmiss�o. Al�m disso, o estado de um objeto, mesmo inalterado, � enviado periodicamente em uma mensagem \textit{set}. Portanto, o protocolo � ideal para ser utilizado em aplicativos controlados interfaces multitoques otimizando a itera��o e confiabilidade da comunica��o.

\textbf{\large{Par�metros das mensagens TUIO}}

\begin{table}[htbp]
  \centering
    \begin{tabular}{|l|l|l|} %rrr}
    \textbf{Parametros} & \textbf{Significado dos parametros} & \textbf{Tipo} \\
    s     & sessionID (ID tempor�rio do objeto) & int32 \\
    x, y  & posi��o & float32 \\
    X, Y  & vetor de movimento (velocidade de movimento e dire��o) & float32 \\
    m     & acelera��o de movimento & float32 \\
    w, h  & largura e altura do \textit{blob} & float32 \\
    \end{tabular}%
		\caption{Par�metros de mensagens TUIO em superf�cies interativas 2D}
  \label{tab:tabela1_param_TUIO}%
\end{table}%

\subsection{\textit{Touchlib}}
\label{cap7.2.3}

\textit{Touchlib}~[\citenum{Touchlib}] � uma biblioteca para auxiliar o processamento de imagens capturadas pela webcam em itera��es com superf�cies multitoque. Ela lida com o acompanhamento de \textit{blobs} de luz infravermelha, e envia para seus programas esses eventos multitoques, como o encostar do dedo, deslocamento do dedo e o retirar do dedo. Inclui um aplicativo de configura��o, algumas demos para exemplificar seu uso. Interage com a maioria dos tipos de webcams e os dispositivos de captura de v�deo sendo muito �til para um projeto de \textit{software} para mesa multitoque. Em contrapartida s� tem suporte para para sistema operacional \textit{Windows}.

Seu funcionamento consiste na aplica��o de filtros de forma customiz�vel como os mostrados nesta figura~[\ref{filtros}]. A execu��o destes filtros s�o realizadas pela \textit{Touchlib} com o uso da \textit{OpenCV (Open Source Computer Vision Library)}, uma biblioteca para o desenvolvimento de aplicativos de processamento de imagens. A execu��o da \textit{Touchlib} funciona de acordo com o paradigma cliente-servidor, sendo a mesma atuando como servidor e a aplica��o como cliente. Esta comunica��o cliente-servidor � feita utilizando o protocolo TUIO~[\citenum{TUIO}] \textit{(Tangible User Interface System)}, permitindo uma arquitetura distribu�da, onde o aplicativo � executado separadamente da aplica��o do processamento de imagens.

\chapter{Conclus�o}
\label{conclusao}
Ao decorrer deste trabalho, conclu�mos que ele ainda n�o acabou e vamos colocar uma conclus�o quando terminar.

\section{Trabalhos Futuros}
\label{cap6.1}
Aqui vamos sugerir o que pode ser feito depois.

\postextual

\bibliographystyle{plain}

\bibliography{bibliografia}

\appendix

\chapter{Refer�ncia N�o Oficial da \textit{Touchlib}}
\label{apendiceA}

A configura��o da Touchlib � toda armazenada em um arquivo \textit{XML}, \textit{config.xml}. Este ap�ndice, adaptado de~[\citenum{Touchlibref}], explica como configurar a Touchlib e como editar o arquivo de configura��o.

\section{Detalhes do Projeto}

\begin{itemize}
	\item \textit{Website} do projeto: www.touchlib.com
	\item \textit{SVN}: http://code.google.com/p/touchlib/
	\item \textit{SVN} (reposit�rio): http://touchlib.googlecode.com/svn/multitouch/
\end{itemize}


\section{Arquivo \textit{config.xml}}
O arquivo de configura��o � dividido nas seguintes partes:
\begin{enumerate}

	\item Defini��o da vers�o do \textit{XML}.
	
	Exemplo: <?xml version="1.0"\ ?>
	
	\item Configura��o do \textit{Tracker}.
	
	Exemplo: <blobconfig distanceThreshold="250"\ minDimension="2"\ maxDimension="250"\ ghostFrames="0"\ minDisplacementThreshold="2.000000"\ />

	Configura��o de toler�ncia do \textit{tracker} de \textit{blobs}. O \textit{distanceThreshold} cont�m o valor de quantos \textit{p�xels} um \textit{blob} pode passar. Os valores de \textit{minDimension} and \textit{maxDimension} especificam o qu�o pequeno ou grande o contorno pode ser para ser detectado como um toque. O valor de \textit{ghostFrames} especif�ca o n�mero de \textit{frames} extras que a \textit{Touchlib} deve usar no \textit{tracker} de \textit{blobs}. O \textit{minDisplacementThreshold} especif�ca quantos \textit{p�xels} de contorno ser�o necess�rios serem movidos antes de se chamar o evento de atualiza��o.
	
	\item Caixa de limita��o.
	
	Exemplo: <bbox ulX="0.000000"\ ulY="0.000000"\ lrX="1.000000"\ lrY="1.000000"\ />
	
	Especif�ca em que regi�o deve ser aplicada a detec��o de blobs.
	
	\item Pontos de calibra��o da tela.
	
	Exemplo: <screen>...[pontos]...</screen>
	
	Estes valores ser�o preenchidos ap�s rodar o configapp.
	
	\item Os gr�ficos de filtro
	
	Exemplo: <filtergraph>...[filtros]...</filtergraph>
	
	Filtros ser�o explicados na pr�xima se��o.
\end{enumerate}
	
\section{Filtros da \textit{Touchlib}}
Todos os filtros precisam estar entre as \textit{tags} de \textit{filtergraph}.
\begin{center}
\textit{<filtergraph> ... </filtergraph>}
\end{center}

O primeiro filtro tem que ser um filtro de captura de v�deo, e o �ltimo deve ser o filtro \textit{rectify}.

\subsection{Filtros de captura de v�deo}
\begin{itemize}
	\item cvccapture.
	
	Descri��o: Filtro de captura de v�deo padr�o usado pela \textit{Touchlib}. Ele usa a fun��o da OpenCV para capturar um \textit{stream} de v�deo. Tamb�m � poss�vel utilizar um arquivo de v�deo para o prop�sito de testes.
	
	Utiliza��o (arquivo de v�deo):
	
	<cvcapture label="cvcapture"\ >
	
	<source value="../tests/rear4.avi"\ />

	</cvcapture>
	
	Utiliza��o (c�mera):
	
	<cvcapture label="cvcapture"\ >
	
	<source value="cam"\ />
	
	</cvcapture>
	
	\item dsvlcapture
	
	Descri��o: Esta implementa��o utiliza \textit{DirectShow} para capturar o \textit{stream} de v�deo.
	
	Utiliza��o:
	
	<dsvlcapture label="dsvlcapture"\ />
	
	\item cmucapture
	
	Descri��o: Este filtro utiliza o driver \textit{CMU} para acessar dispositivos de captura de v�deo \textit{FireWire}.
	
	Utiliza��o:
	
	<cmucapture label="cmucapture"\ >
	
	<brightness value="-1"\ />
	
	<exposure value="-1"\ />
	
	<flipRGB value="false"\ />
	
	<gain value="87"\ />
	
	<gamma value="1"\ />
	
	<mode value="640x480mono"\ />
	
	<rate value="30fps"\ />
	
	<saturation value="90"\ />
	
	<sharpness value="80"\ />
	
	<whitebalanceH value="0"\ />
	
	<whitebalanceL value="-1"\ />
	
	</cmucapture>

	\item vwcapture
	
	Descri��o: Este filtro utiliza a \textit{API VideoWrapper} para acessar dispositivos de captura de v�deo \textit{FireWire}.
	
	Utiliza��o:
	
	<vwcapture label="capture1"\ >
	
	<videostring value="pgr: 0 640 30 grey16 1 rgb"\ />
	
	</vwcapture>
	
	O \textit{videostring} cont�m os par�metros da c�mera que dependem do tipo de interface:
	
	
\begin{itemize}
	\item C�meras utilizando a especifica��o \textit{DCAM} do \textit{Videre Design}.
	
	Par�metros: "dcam: n�mero largura \textit{frameRate} modoDeCor escala"\ 
	
	Exemplo: "dcam: 0 640 30 rgb 2"\ 
	
	
	
	\item C�meras do \textit{Point Grey Research}.
	
	Par�metros: "pgr: n�mero largura \textit{frameRate} modoDeCor escala modoDaSa�da"\ 
	
	Exemplo: "pgr: 0 640 30 grey8 1 rgb"\ 
	
	
	
	\item C�meras utilizando \textit{VidCapture} (\textit{DirectShow}).
	
	Par�metros: "vc: n�mero largura \textit{frameRate} modoDeCor escala modoDaSa�da"\ 
	
	Exemplo: "vc: 0 640 15 rgb 0"\ 
\end{itemize}
	
	
\end{itemize}

\subsection{Filtros de processamento de v�deo}

\begin{itemize}
	\item Filtro de \textit{Mono}.
	
	Descri��o: A \textit{Touchlib} requer uma imagem fonte de 8 \textit{bits} em escala de cinza. Este filtro s� � necess�rio se o filtro de captura de v�deo n�o for capaz de entregar o formato correto.
	
	Utiliza��o:
	
	<mono label="monofilter"\ />
	
	\item Filtro de \textit{Background}.
	
	Descri��o: Este filtro remove o fundo atrav�s da cria��o de uma imagem de refer�ncia na sua inicializa��o, que depois � subtraida do frame ativo atual.
	
	Utiliza��o:
	
	<backgroundremove label="backgroundfilter"\ >
	
	<threshold value="20"\ />
	
	</backgroundremove>
	
	o valor de \textit{threshold} varia de 0-255.
	
	\item Filtro de \textit{Smoothing}.
	
	Descri��o: Este filtro de suaviza��o aplica o \textit{gaussian blur} na imagem fonte.
	
	Utiliza��o:
	
	<smooth label="smoothfilter"\ />
	
	\item Filtro de \textit{Invert}.
	
	Descri��o: Este filtro inverte uma imagem em escala de cinza. Ele s� � necess�rio em caso de detec��o reversa, onde se detecta os pontos que n�o est�o iluminados.
	
	Utiliza��o:
	
	<invert label="invert"\ />
	
	\item Filtro de \textit{Scaler}.
	
	Descri��o: Se o filtro usado antes deste der uma sa�da fraca, este filtro de escala � utilizado para ampliar a imagem atual.
	
	Utiliza��o:
	
	<scaler label="scaler"\ >
	
	<level value="70"\ />
	
	</scaler>
	
	\item Filtro de Brilho e Contraste.
	
	Descri��o: Situa��es particulares podem precisar de um ajuste no brilho e contraste da imagem. Este filtro normalmente � usado para ilumina��es FTIR.
	
	Utiliza��o:
	
	<brightnesscontrast label="brightnesscontrast4"\ >
	
	<brightness value="0.1"\ />
	
	<contrast value="0.4"\ />
	
	</brightnesscontrast>
	
	\item Filtro de \textit{Highpass}.
	
	Descri��o: Ilumina��es difusas produzem um \textit{blob} com brilho muito inferior a ilumina��o FTIR. No entanto, se tiver contraste suficiente dispon�vel, o filtro de \textit{Highpass} � capaz de aplificar a sa�da desses \textit{blobs}.
	
	Utiliza��o:
	
	<highpass label="highpass"\ >
	
	<filter value="6"\ />
	
	<scale value="32"\ />
	
	</highpass>
	
	\item Filtro de \textit{Highpass} (simples).
	
	Descri��o: Similar ao filtro anterior, mas usa um algoritmo mais simples, possuindo uma performance mais r�pida que o filtro de \textit{Highpass} padr�o. Este filtro possui duas maneiras diferentes de amplificar a imagem fonte, � recomendado utilizar o \textit{noiseMethod} como 1.
	
	Utiliza��o:
	
	<simplehighpass label="simplehighpass"\ >
	
	<blur value="13"\ />
	
	<noise value="3"\ />
	
	<noiseMethod value="1"\ />
	
	</simplehighpass>
	
	\item Filtro de Corre��o da Distor��o em Barril.
	
	Descri��o: Uma lente de c�mera de �ngulo grande normalmente cont�m uma distor��o radial que n�o pode ser corrigida pelo algoritmo padr�o da \textit{Touchlib}. Este filtro corrige a distor��o em barril baseada na caracter�stica da lente. Este filtro necessita de um arquivo \textit{camera.yml} que � criado pela ferramenta de corre��o da distor��o em barril. Como a imagem � corrigida, o resultado pode perder algumas partes importantes. Pode-se arrumar isso mudando o \textit{border size} (se n�o for necess�rio, este valor deve ser colocado como 0).
	
	Utiliza��o:
	
	<barreldistortioncorrection label="barreldistortioncorrection1"\ >
	
	<border size value="20"\ />
	
	</barreldistortioncorrection>
	
	\item Filtro de Corte.
	
	Descri��o: Permite o usu�rio a cortar a imagem do v�deo. Os valores \textit{posX} e \textit{posY} definem a posi��o de cima e esquerda que come�a a �rea de corte. O \textit{height} e \textit{width} definem a altura e largura, respectivamente, da �rea de corte.
	
	Utiliza��o:
	
	<crop label="crop"\ >
	
	<posX value="40"\ />
	
	<posY value="40"\ />
	
	<height value="120"\ />
	
	<width value="160"\ />
	
	</crop>
	
	\item Filtro \textit{Rectify}.
	
	Descri��o: Este � o filtro final do \textit{pipeline} do processamento de imagem. Ele deve ser colocado em um valor no qual os \textit{blobs} sejam vis�veis mas a entrada de barulho seja ignorada. Esta imagem ser� a usada para se fazer a detec��o de \textit{blobs}.
	
	Utiliza��o:
	
	<rectify label="rectify"\ >
	
	<level value="75"\ />
	
	</rectify>
	
	O valor de \textit{level} varia de 0-255.
	
\end{itemize}

\section{Calibra��o da \textit{Touchlib}}

Para calibrar a Touchlib, � necess�rio possuir uma mesa multi-toque completamente funcional, incluindo projetor e c�mera. � recomendado que utilize arquivos de \textit{config.xml} de exemplo para poder criar um. Cada t�cnica de ilumina��o requer uma cadeia diferente de filtros. Quando a cadeia tiver finalizada, a aplica��o de configura��o deve ser usada iniciando o arquivo \textit{configapp.exe} do diret�rio \textit{bin} da \textit{Touchlib} (ou \textit{./configapp} do diret�rio \textit{/src} no \textit{Linux}). A aplica��o agora ir� mostrar os filtros iniciados no arquivo \textit{config.xml}. Aperte 'b' para capturar a imagem de refer�ncia do filtro de \textit{background}, depois ajuste os par�metros at� que a janela do filtro \textit{rectify} s� mostre \textit{blobs} claros quando a mesa for tocada. Se for necess�rio recapturar a imagem de refer�ncia para o filtro de \textit{background}, aperte 'b'.

Quando tiver satisfeito com os resultados da cadeia de filtros, inicie a calibra��o apertando 'enter' para ir para o modo de calibra��o em tela cheia. A aplica��o de configura��o ir� agora mostrar um fundo preto com 20 cruzes verdes de refer�ncia, 5 horizontais por linha e 4 verticais. No canto superior esquerdo tem um exemplo do input da c�mera. Pode-se mudar o filtro que se v� no exemplo apertado os n�meros de 1-0. Se n�o quiser utilizar o tamanho completo do v�deo da c�mera, � poss�vel apertar 'x' para ajustar a caixa de limita��o, seguindo as instru��es na tela. Para come�ar a calibra��o, aperte 'c'. O ponto de refer�ncia atual a ser selecionado fica marcado em vermelho. Se a sua c�mera estiver em alguma orienta��o diferente, a \textit{Touchlib} ir� corrigir isso automaticamente durante a calibra��o, ao tocar no primeiro ponto. Continue a calibra��o tocando nos pontos restantes. Ao terminar a calibra��o, teste a precis�o dela tocando na tela. A aplica��o mostrar� um ret�ngulo do tamanho do toque no local em que for tocado a superf�cie. Para sair da aplica��o de configura��o, aperte 'ESC'. Ao sair, todos os valores da aplica��o s�o escritos para o arquivo \textit{config.xml}. Se necess�rio, pode se ajustar os valores da calibra��o dentro das \textit{tags <screen>...</screen>}.
\chapter{Configura��o da \textit{Touchlib}}
\label{apendiceB}

O arquivo \textit{config.xml} feito ao decorrer deste trabalho segue abaixo.


\lstinputlisting[language=XML,basicstyle=\ttdefault]{touchlibconfig.xml}

\end{document}


